\documentclass[10pt,a4paper]{report}
\usepackage[utf8]{inputenc}
\usepackage{amsmath}
\usepackage{amsfonts}
\usepackage{amssymb}
\usepackage{graphicx}
\usepackage{hyperref}
\usepackage{amsthm}
\usepackage{enumitem}
\usepackage{xeCJK}
\usepackage{extarrows}

\def\*#1{\mathbf{#1}}
\def\rand{\xleftarrow{\$}}

\title{}
\author{}
\date{}
\begin{document}

\chapter*{第三章:行列式}
\noindent 1.
\begin{proof}[解]
	根据行列式的定义当$(i_{1}i_{2}i_{3}i_{4}) = (1234)$时可以构成$x^{4}$\\
	$$(-1)^{\tau(1234)}a_{11}a_{22}a_{33}a_{44} = 2x^{4}$$\\
	所以,$x^{4}$的系数为2.\\
	同理,当$(i_{1}i_{2}i_{3}i_{4}) = (2134)$时可以构成$x^{3}$
	$$(-1)^{\tau(2134)}a_{21}a_{12}a_{33}a_{44} = -x^{3}$$\\
	所以,$x^{4}$的系数为-1.\\
\end{proof}
\noindent 2.
\begin{proof}[解]
	将行列式按定义展开每一项是
	$$(-1)^{\tau(i_{1}i_{2}i_{3}i_{4}i_{5})}(a_{i_{1}1}a_{i_{2}2}a_{i_{3}3}a_{i_{4}4}a_{i_{5}5})$$
	而且$a_{i_{3}3},a_{i_{4}4},a_{i_{5}5}$ 至少有一个为0.\\
	所以行列式展开每一项都是0.行列式值为0.
\end{proof}
\noindent 3.
\begin{proof}[解]
	如果n阶行列式中的0的个数比$n^{2}-n$多,那么至少存在一行(列)全为0.所以行列式必等于0.
\end{proof}
\noindent 4.
\begin{proof}[解]
	$$D = \left|
	\begin{matrix}
	1 & 1 & \cdots & 1 & 1\\
	1 & 1 & \cdots & 1 & 1\\
	\vdots & \vdots & \ddots & \vdots & \vdots\\
	1 & 1 & \cdots & 1 & 1 \\
	1 & 1 & \cdots & 1 & 1 
	\end{matrix}
	\right| = \sum_{i_{1}i_{2} \cdots i_{n}}(-1)^{\tau(i_{1}i_{2} \cdots i_{n})} = 0\\$$
	奇排列下每一项值为$-1$,偶排列下为$1$,所以令奇排列有$n$个,偶排列$m$个,则$m-n = 0, m=n$\\
	所以奇偶排列各半.
\end{proof}
\noindent 5.
\begin{proof}[解]
	(1)$D=0$,因为第二列为$0$.\\
	(2)$D=\left|
	\begin{matrix}
	1 & 1 & 1 & 3\\
	0 & 3 & 1 & 1\\
	0 & 0 & 2 & 2 \\
	0 & 0 & 0 & 5 
	\end{matrix}
	\right| = 30$
\end{proof}
\noindent 6.
\begin{proof}[解]
	(1)当$x \neq 0$时\\
	$$D_{2n} = \left|
	\begin{matrix}
	x & 0 & \cdots & 0 & y\\
	0 & x & \cdots & y & 0\\
	\vdots & \vdots & \ddots & \vdots & \vdots\\
	0 & y & \cdots & x & 0 \\
	y & 0 & \cdots & 0 & x 
	\end{matrix}
	\right|\\$$
	$$
	=\left|
	\begin{matrix}
	x & 0 & \cdots & 0 & y\\
	0 & x & \cdots & y & 0\\
	\vdots & \vdots & \ddots & \vdots & \vdots\\
	0 & 0 & \cdots & x-y^{2}/x & 0 \\
	0 & 0 & \cdots & 0 & x-y^{2}/x 
	\end{matrix}
	\right|\\
	$$
	$$
	=x^{n}(x-y^{2}/x)^{n} = (x^{2}-y^{2})^{n}
	$$
	(2)当$x = 0$时
	$$D_{2n} = \left|
	\begin{matrix}
	0 & 0 & \cdots & 0 & y\\
	0 & 0 & \cdots & y & 0\\
	\vdots & \vdots & \ddots & \vdots & \vdots\\
	0 & y & \cdots & 0 & 0 \\
	y & 0 & \cdots & 0 & 0 
	\end{matrix}
	\right|\\$$
	$$
	=(-1)^{\tau(2n,2n-1,...,2,1)}y^{2n} = (-1)^{2n(2n-1)/2}y^{2n} = (-y^{2})^{n}
	$$
\end{proof}
\noindent 7.
\begin{proof}[解]
	$$
	D_{n} = \left|
	\begin{matrix}
	x & y & 0 & \cdots & 0 & 0\\
	0 & x & y & \cdots & 0 & 0\\
	0 & 0 & x & \cdots & 0 & 0\\
	\vdots & \vdots & \vdots & \ddots & \vdots & \vdots\\
	0 & 0 & 0 & \cdots & x & y \\
	y & 0 & 0 & \cdots & 0 & x 
	\end{matrix}
	\right|\\
	$$
	$$
	= x\left|
	\begin{matrix}
	x & y & \cdots & 0 & 0\\
	0 & x & \cdots & 0 & 0\\
	\vdots & \vdots & \ddots & \vdots & \vdots\\
	0 & 0 & \cdots & x & y \\
	0 & 0 & \cdots & 0 & x 
	\end{matrix}
	\right|
	+(-1)^{1+n}y\left|
	\begin{matrix}
	y & 0 & \cdots & 0 & 0\\
	x & y & \cdots & 0 & 0\\
	0 & x & \cdots & 0 & 0\\
	\vdots & \vdots & \ddots & \vdots & \vdots\\
	0 & 0 & \cdots & x & y  
	\end{matrix}
	\right|
	$$
	$$
	=x^{n}+(-1)^{1+n}y^{n}
	$$
\end{proof}
\noindent 8.
\begin{proof}[解]
	$$
	D_{n} = \left|
	\begin{matrix}
	2 & 1 & 0 & \cdots & 0 & 0 & 0\\
	1 & 2 & 1 & \cdots & 0 & 0 & 0\\
	0 & 1 & 2 & \cdots & 0 & 0 & 0\\
	\vdots & \vdots & \vdots & \ddots & \vdots & \vdots & \vdots \\
	0 & 0 & 0 & \cdots & 2 & 1 & 0 \\
	0 & 0 & 0 & \cdots & 1 & 2 & 1\\
	0 & 0 & 0 & \cdots & 0 & 1 & 2
	\end{matrix}
	\right|\\
	$$
	$$
	=2\left|
	\begin{matrix}
	2 & 1 & \cdots & 0 & 0 & 0\\
	1 & 2 & \cdots & 0 & 0 & 0\\
	\vdots & \vdots & \ddots &\vdots & \vdots & \vdots \\
	0 & 0 & \cdots & 2 & 1 & 0 \\
	0 & 0 & \cdots & 1 & 2 & 1\\
	0 & 0 & \cdots & 0 & 1 & 2
	\end{matrix}
	\right|
	-\left|
	\begin{matrix}
	1 & 0 & \cdots & 0 & 0 & 0\\
	1 & 2 & \cdots & 0 & 0 & 0\\
	\vdots & \vdots & \ddots & \vdots & \vdots & \vdots \\
	0 & 0 & \cdots & 2 & 1 & 0 \\
	0 & 0 & \cdots & 1 & 2 & 1\\
	0 & 0 & \cdots & 0 & 1 & 2
	\end{matrix}
	\right|
	$$
	$$
	=2D_{n-1}-D_{n-2}
	$$
	$$
	D_{n}-D_{n-1} = D_{n-1}-D_{n-2} = ... = D_{2}-D_{1} = 3-2 = 1
	$$
	$$
	D_{n}-D_{n-1} + D_{n-1}-D_{n-2} + ... + D_{2}-D_{1} = n-1
	$$
	$$
	D_{n} = n-1+D_{1} = n-1+2 = n+1
	$$
\end{proof}
\noindent 9.
\begin{proof}[解]
	将行列式按第一列展开:\\
	$B_{n} = 
	\left|
	\begin{matrix}
	a+b & ab & 0 & \cdots & 0 & 0 & 0 & 0\\
	1 & a+b & ab & \cdots & 0 & 0 & 0 & 0\\
	\vdots & \vdots & \vdots & \ddots & \vdots & \vdots & \vdots & \vdots \\
	0 & 0 & 0 & \cdots & 1 & a+b & ab & 0 \\
	0 & 0 & 0 & \cdots & 0 & 1 & a+b & ab\\
	0 & 0 & 0 & \cdots & 0 & 0 & 1 & a+b
	\end{matrix}
	\right|\\
	(a+b)
	\left|
	\begin{matrix}
	a+b & ab & \cdots & 0 & 0 & 0 & 0\\
	\vdots & \vdots & \ddots & \vdots & \vdots & \vdots & \vdots \\
	0 & 0 & \cdots & 1 & a+b & ab & 0 \\
	0 & 0 & \cdots & 0 & 1 & a+b & ab\\
	0 & 0 & \cdots & 0 & 0 & 1 & a+b
	\end{matrix}
	\right|\\
	+(-1)^{1+2}
	\left|
	\begin{matrix}
	ab & 0 & \cdots & 0 & 0 & 0 & 0\\
	\vdots & \vdots & \ddots & \vdots & \vdots & \vdots & \vdots \\
	0 & 0 & \cdots & 1 & a+b & ab & 0 \\
	0 & 0 & \cdots & 0 & 1 & a+b & ab\\
	0 & 0 & \cdots & 0 & 0 & 1 & a+b
	\end{matrix}
	\right|\\
	=(a+b)B_{n-1}-abB_{n-2}$\\
	我们可以得到:\\
	$B_{n}-bB_{n-1} = aB_{n-1}-abB_{n-2} = a(B_{n-1}-bB_{n-2})$\\
	$B_{1} = a+b$,
	$B_{2} = \left|
	\begin{matrix}
	a+b & ab\\
	1 & a+b 
	\end{matrix}
	\right| = a^{2}+ab+b^{2}$\\
	$B_{2}-bB_{1} = a^{2}+ab+b^{2}-ab-b^{2}=a^{2}$\\
	所以$B_{n}-bB_{n-1} = a^{n-2}a^{2} = a^{n}$\\
	同理可得$B_{n}-aB_{n-1} = b^{n-2}b^{2} = b^{n}$\\
	当$a \neq b$时:\\
	得到$aB_{n}-bB_{n} = a^{n+1}-b^{n+1}$\\
	即$B_{n} = \frac{a^{n+1}-b^{n+1}}{a-b}$\\
	当$a = b$时:\\
	得到$B_{n}-aB_{n-1} = a^{n}$
	即$\frac{B_{n}}{a^{n}} - \frac{B_{n-1}}{a^{n-1}} = 1$\\
	又$\frac{B_{1}}{a} = 2$, 所以$\frac{B_{n}}{a^{n}} = n+1$, 即$B_{n} = (n+1)a^{n}$.
\end{proof}
\noindent 10.
\begin{proof}[解]
	$$D_{n} = \left|
	\begin{matrix}
	2cos\theta & 1 & 0 & \cdots & 0 & 0 & 0\\
	1 & 2cos\theta & 1 & \cdots & 0 & 0 & 0\\
	0 & 1 & 2cos\theta & \cdots & 0 & 0 & 0\\
	\vdots & \vdots & \vdots & \ddots & \vdots & \vdots & \vdots \\
	0 & 0 & 0 & \cdots & 2cos\theta & 1 & 0 \\
	0 & 0 & 0 & \cdots & 1 & 2cos\theta & 1\\
	0 & 0 & 0 & \cdots & 0 & 1 & 2cos\theta
	\end{matrix}
	\right|$$
	$$
	= 2cos\theta D_{n-1}-D_{n-2}
	$$
	$$
	D_{1} = 2cos\theta = \frac{sin2\theta}{sin\theta}
	$$
	$$
	D_{2} = 4cos^{2}\theta-1 = \frac{sin3\theta}{sin\theta}
	$$
	$$
	D_{3} = 8cos^{3}\theta-4cos\theta = \frac{sin4\theta}{sin\theta}
	$$
	我们大胆猜想\\
	$$
	D_{n} = \frac{sin(n+1)\theta}{sin\theta}
	$$
	用数学归纳法证明\\
	当$n = 1, 2, 3$时显然成立.\\
	假设$D_{n-2} = \frac{sin(n-1)\theta}{sin\theta}, D_{n-1} = \frac{sinn\theta}{sin\theta}$.\\
	则
	$$
	D_{n} = 2cos\theta D_{n-1}-D_{n-2} = 2cos\theta \frac{sinn\theta}{sin\theta}-\frac{sin(n-1)\theta}{sin\theta}
	$$
	$$
	= \frac{cos\theta sinn\theta + cosn\theta sin\theta + cos\theta sinn\theta - cosn\theta sin\theta - sin(n-1)\theta}{sin\theta}
	$$
	$$
	= \frac{sin(n+1)\theta+sin(n-1)\theta-sin(n-1)\theta}{sin\theta} = \frac{sin(n+1)\theta}{sin\theta}
	$$
	所以,假设成立.
\end{proof}
\noindent 17.
\begin{proof}[解]
	当$n = 1$时\\
	$$D_{1} = a_{1}+b_{1}$$
	当$n = 2$时\\
	$$D_{2} = a_{1}b_{2}+a_{2}b_{1}-a_{1}b_{1}-a_{2}b_{2}$$
	当$n \geq 3$时\\
	$$D_{n}= \left|
	\begin{matrix}
	a_{1}+b_{1} & a_{1}+b_{2} & \cdots & a_{1}+b_{n}\\
	a_{2}+b_{1} & a_{2}+b_{2} & \cdots & a_{2}+b_{n}\\
	\vdots & \vdots & \ddots & \vdots\\
	a_{n}+b_{1} & a_{n}+b_{2} & \cdots & a_{n}+b_{n}
	\end{matrix}
	\right|$$
	$$
	=\left|
	\begin{matrix}
	a_{1} & a_{1} & \cdots & a_{1}\\
	a_{2}+b_{1} & a_{2}+b_{2} & \cdots & a_{2}+b_{n}\\
	\vdots & \vdots & \ddots & \vdots\\
	a_{n}+b_{1} & a_{n}+b_{2} & \cdots & a_{n}+b_{n}
	\end{matrix}
	\right|+\left|
	\begin{matrix}
	b_{1} & b_{2} & \cdots & b_{n}\\
	a_{2}+b_{1} & a_{2}+b_{2} & \cdots & a_{2}+b_{n}\\
	\vdots & \vdots & \ddots & \vdots\\
	a_{n}+b_{1} & a_{n}+b_{2} & \cdots & a_{n}+b_{n}
	\end{matrix}
	\right|
	$$
	$$
	=\left|
	\begin{matrix}
	a_{1} & 0 & \cdots & 0\\
	a_{2}+b_{1} & b_{2}-b_{1} & \cdots & b_{n}-b_{1}\\
	\vdots & \vdots & \ddots & \vdots\\
	a_{n}+b_{1} & b_{2}-b_{1} & \cdots & b_{n}-b_{1}
	\end{matrix}
	\right|+\left|
	\begin{matrix}
	b_{1} & b_{2} & \cdots & b_{n}\\
	a_{2} & a_{2} & \cdots & a_{2}\\
	\vdots & \vdots & \ddots & \vdots\\
	a_{n} & a_{n} & \cdots & a_{n}
	\end{matrix}
	\right|
	$$
	$$
	a_{1}\left|
	\begin{matrix}
	b_{2}-b_{1} & \cdots & b_{n}-b_{1}\\
	\vdots & \ddots & \vdots\\
	b_{2}-b_{1} & \cdots & b_{n}-b_{1}
	\end{matrix}
	\right|+0=0
	$$
\end{proof}
\noindent 27.
\begin{proof}[解]
	当$n=1$时\\
	$$
	D_{1} = 1+x_{1}y_{1}
	$$
	当$n=2$时\\
	$$
	D_{2} = x_{1}y_{1}+x_{2}y_{2}-x_{1}y_{2}-x_{2}y_{1}
	$$
	当$n \geq 3$时\\
	$$D_{n}= \left|
	\begin{matrix}
	1+x_{1}y_{1} & 1+x_{1}y_{2} & \cdots & 1+x_{1}y_{n}\\
	1+x_{2}y_{1} & 1+x_{2}y_{2} & \cdots & 1+x_{2}y_{n}\\
	\vdots & \vdots & \ddots & \vdots\\
	1+x_{n}y_{1} & 1+x_{n}y_{2} & \cdots & 1+x_{n}y_{n}
	\end{matrix}
	\right|$$
	$$
	= \left|
	\begin{matrix}
	1 & 1+x_{1}y_{2} & \cdots & 1+x_{1}y_{n}\\
	1 & 1+x_{2}y_{2} & \cdots & 1+x_{2}y_{n}\\
	\vdots & \vdots & \ddots & \vdots\\
	1 & 1+x_{n}y_{2} & \cdots & 1+x_{n}y_{n}
	\end{matrix}
	\right|+\left|
	\begin{matrix}
	x_{1}y_{1} & 1+x_{1}y_{2} & \cdots & 1+x_{1}y_{n}\\
	x_{2}y_{1} & 1+x_{2}y_{2} & \cdots & 1+x_{2}y_{n}\\
	\vdots & \vdots & \ddots & \vdots\\
	x_{n}y_{1} & 1+x_{n}y_{2} & \cdots & 1+x_{n}y_{n}
	\end{matrix}
	\right|
	$$
	$$
	\left|
	\begin{matrix}
	1 & x_{1}y_{2} & \cdots & x_{1}y_{n}\\
	1 & x_{2}y_{2} & \cdots & x_{2}y_{n}\\
	\vdots & \vdots & \ddots & \vdots\\
	1 & x_{n}y_{2} & \cdots & x_{n}y_{n}
	\end{matrix}
	\right|+\left|
	\begin{matrix}
	x_{1}y_{1} & 1 & \cdots & 1\\
	x_{2}y_{1} & 1 & \cdots & 1\\
	\vdots & \vdots & \ddots & \vdots\\
	x_{n}y_{1} & 1 & \cdots & 1
	\end{matrix}
	\right|
	$$
	$$
	=0+0=0
	$$
\end{proof}
\noindent 28(3).
\begin{proof}[解]
	$$A^{-1} = A^{*}/|A| = A^{*} = \left[
	\begin{matrix}
	1 & 0 & -1 & -1\\
	0 & 1 & 0 & 0\\
	0 & 0 & -1 & -1 \\
	0 & 0 & 0 & -1 
	\end{matrix}
	\right]
	$$
\end{proof}
\noindent 29.
\begin{proof}[解]
	$AA^{*}=|A|E$\\
	$A$可逆,则$A^{*}=|A|A^{-1}$\\
	所以,$|(\frac{1}{4}A)^{-1}-2A^{*}|\\
	=|4A^{-1} - 2A^{*}|\\
	=|4A^{-1} - 2|A|A^{-1}|\\
	=|(4-2|A|)A^{-1}|\\
	=|(-2)A^{-1}|\\
	=(-2)^{n}|A|^{-1}\\
	=\frac{(-2)^{n}}{3}
	$
\end{proof}
\noindent 30.
\begin{proof}[解]
	$ABA^{-1}=BA^{-1}+3I$\\
	$\Rightarrow AB = B+3A$\\
	$\Rightarrow (A-I)B=3A$\\
	$\Rightarrow (I-A^{-1})B=3I$\\
	$|A^{*}| = ||A|A^{-1}|=|A|^{n-1}$\\
	$|A| = |A^{*}|^{\frac{1}{n-1}} = |A^{*}|^{\frac{1}{3}} = 2$\\
	$I-A^{-1} = I-\frac{A^{*}}{|A|} = \left[
	\begin{matrix}
	1/2 & 0 & 0 & 0\\
	0 & 1/2 & 0 & 0\\
	-1/2 & 0 & 1/2 & 0\\
	0 & 3/2 & 0 & -3
	\end{matrix}
	\right]$可逆\\
	所以,$B=3(I-A^{-1})^{-1} = \left[
	\begin{matrix}
	6 & 0 & 0 & 0\\
	0 & 6 & 0 & 0\\
	6 & 0 & 6 & 0\\
	0 & 3 & 0 & -1
	\end{matrix}
	\right]$
\end{proof}
\noindent 34.
\begin{proof}[解]
	(1)(2)的秩都是2.
\end{proof}
\end{document}