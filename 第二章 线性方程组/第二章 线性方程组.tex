\documentclass[10pt,a4paper]{report}
\usepackage[utf8]{inputenc}
\usepackage{amsmath}
\usepackage{amsfonts}
\usepackage{amssymb}
\usepackage{graphicx}
\usepackage{hyperref}
\usepackage{amsthm}
\usepackage{enumitem}
\usepackage{xeCJK}

\def\*#1{\mathbf{#1}}
\def\rand{\xleftarrow{\$}}

\title{}
\author{}
\date{}
\begin{document}

\chapter*{第二章:线性方程组}
\noindent 2.解下列线性方程组:\\
\noindent (1)
$$
\left\{
\begin{aligned}
x_{1}-2x_{2}+3x_{3}-4x_{4}=0 \\
x_{2}-x_{3}+x_{4}=0 \\
x_{1}+3x_{2}-3x_{4}=0 \\
x_{1}-4x_{2}+3x_{3}-2x_{4}=0
\end{aligned}
\right.
$$
\begin{proof}[解]
	$
	\left[
	\begin{matrix}
	1 & -2 & 3 & -4\\
	0 & 1 & -1 & 1\\
	1 & 3 & 0 & -3 \\
	1 & -4 & 3 & -2
	\end{matrix}
	\right] \xrightarrow[R_{4}-R_{1};\frac{1}{2}R_{4}]{R_{3} -R_{1}} \left[
	\begin{matrix}
	1 & -2 & 3 & -4\\
	0 & 1 & -1 & 1\\
	0 & 5 & -3 & 1\\
	0 & -1 & 0 & 1
	\end{matrix}
	\right] \xrightarrow[R_{4}+R_{2}]{R_{3}-5R_{2}} \\
	\left[
	\begin{matrix}
	1 & -2 & 3 & -4\\
	0 & 1 & -1 & 1\\
	0 & 0 & 1 & -2\\
	0 & 0 & -1 & 2
	\end{matrix}
	\right] \xrightarrow[R_{2}+R_{3}]{R_{4}+R_{3}} \left[
	\begin{matrix}
	1 & -2 & 3 & -4\\
	0 & 1 & 0 & -1\\
	0 & 0 & 1 & -2\\
	0 & 0 & 0 & 0
	\end{matrix}
	\right] \xrightarrow{R_{1}+2R_{2}-3R_{3}} \left[
	\begin{matrix}
	1 & 0 & 0 & 0\\
	0 & 1 & 0 & -1\\
	0 & 0 & 1 & -2\\
	0 & 0 & 0 & 0
	\end{matrix}
	\right]
	$
	把$x_{1},x_{2},x_{3}$作为主元变量,$x_{4}$作为自由变量,解得
	$$
	\left\{
	\begin{aligned}
	x_{1} = & 0 \\
	x_{2} = & t_{4} \\
	x_{3} = & 2t_{4} \\
	x_{4} = & t_{4}
	\end{aligned}
	\right.
	$$
\end{proof}
\noindent (2)
$$
\left\{
\begin{aligned}
x_{1}+x_{2}+x_{3}+x_{4}+x_{5} = 0 \\
3x_{1}+2x_{2}+x_{3}+x_{4}-3x_{5}=0 \\
x_{2}+2x_{3}+2x_{4}+6x_{5}=0 \\
5x_{1}+4x_{2}+3x_{3}+3x_{4}-x_{5}=0
\end{aligned}
\right.
$$
\begin{proof}[解]
	$
	\left[
	\begin{matrix}
	1 & 1 & 1 & 1 & 1\\
	3 & 2 & 1 & 1 & -3\\
	0 & 1 & 2 & 2 & 6 \\
	5 & 4 & 3 & 3 & -1
	\end{matrix}
	\right] \xrightarrow[R_{4}-5R_{1}]{R_{2}-3R_{1}}
	\left[
	\begin{matrix}
	1 & 1 & 1 & 1 & 1\\
	0 & -1 & -2 & -2 & -6\\
	0 & 1 & 2 & 2 & 6 \\
	0 & -1 & -2 & -2 & -6
	\end{matrix}
	\right]\\ \xrightarrow[R_{3}+R_{2}]{R_{4}-R_{2}}
	\left[
	\begin{matrix}
	1 & 1 & 1 & 1 & 1\\
	0 & -1 & -2 & -2 & -6\\
	0 & 0 & 0 & 0 & 0 \\
	0 & 0 & 0 & 0 & 0
	\end{matrix}
	\right] \xrightarrow{R_{1}+R_{2};-R_{2}}
	\left[
	\begin{matrix}
	1 & 0 & -1 & -1 & -5\\
	0 & 1 & 2 & 2 & 6\\
	0 & 0 & 0 & 0 & 0 \\
	0 & 0 & 0 & 0 & 0
	\end{matrix}
	\right]\\
	$
	把$x_{1},x_{2}$作为主元变量,$x_{3},x_{4},x_{5}$作为自由变量,解得
	$$
	\left\{
	\begin{aligned}
	x_{1} = & t_{3}+t_{4}+5t_{5} \\
	x_{2} = & -2t_{3}-2t_{4}-6t_{5} \\
	x_{3} = & t_{3} \\
	x_{4} = & t_{4} \\
	x_{5} = & t_{5}
	\end{aligned}
	\right.
	$$
\end{proof}
\noindent (3)
$$
\left\{
\begin{aligned}
x_{1}+2x_{2}+x_{3}-x_{4}=6\\
2x_{1}-x_{2}+x_{3}+3x_{4}+4x_{5}=-7 \\
2x_{1}-x_{2}+2x_{3}+x_{4}-2x_{5}=-4 \\
2x_{1}-3x_{2}+x_{3}+2x_{4}-2x_{5}=-9 \\
x_{1}+x_{3}-2x_{4}-6x_{5} = 4
\end{aligned}
\right.
$$
\begin{proof}[解]
	$$
	\left[
	\begin{matrix}
	1 & 2 & 1 & -1 & 0 & 6\\
	2 & -1 & 1 & 3 & 4 & -7\\
	2 & -1 & 2 & 1 & -2 & -4\\
	2 & -3 & 1 & 2 & -2 & -9 \\
	1 & 0 & 1 & -2 & -6 & 4
	\end{matrix}
	\right]
	$$
	$$
	\xrightarrow{...}
	\left[
	\begin{matrix}
	1 & 0 & 0 & 0 & 0 & 1\\
	0 & 1 & 0 & 0 & 20/11 & 2\\
	0 & 0 & 1 & 0 & -14/11 & -1\\
	0 & 0 & 0 & 1 & 26/11 & -2 \\
	0 & 0 & 0 & 0 & 0 & 0
	\end{matrix}
	\right]
	$$
	把$x_{1},x_{2},x_{3},x_{4}$作为主元变量,$x_{5}$作为自由变量,解得
	$$
	\left\{
	\begin{aligned}
	x_{1} = & 1 \\
	x_{2} = & 2-\frac{20}{11}t_{5} \\
	x_{3} = & -1+\frac{14}{11}t_{5} \\
	x_{4} = & -2-\frac{26}{11}t_{5} \\
	x_{5} = & t_{5}
	\end{aligned}
	\right.
	$$
\end{proof}
\noindent (4)
$$
\left\{
\begin{aligned}
2x_{1}-x_{2}-2x_{3}+x_{4}=0\\
1x_{1}+2x_{2}+2x_{3}+x_{4}=6\\
3x_{1}+x_{2}-x_{3}-2x_{4}=1\\
x_{1}+2x_{2}+x_{3}-3x_{4}=2\\
2x_{1}+4x_{2}+3x_{3}-2x_{4}=7
\end{aligned}
\right.
$$
\begin{proof}[解]
	$$
	\left[
	\begin{matrix}
	2 & -1 & -2 & 1 & 0\\
	1 & 2 & 2 & 1 & 6\\
	3 & 1 & -1 & -2 & 1\\
	1 & 2 & 1 & -3 & 2\\
	2 & 4 & 3 & -2 & 7
	\end{matrix}
	\right]
	$$
	$$
	\xrightarrow{...}
	\left[
	\begin{matrix}
	1 & 2 & 2 & 1 & 6\\
	0 & -5 & -6 & -1 & -12\\
	0 & 0 & -1 & -4 & -5\\
    0 & 0 & 0 & 0 & 1\\
	0 & 0 & 0 & 0 & 0
	\end{matrix}
	\right]
	$$
	$r(A) < r(A|b)$,所以方程组无解.
\end{proof}
\noindent 4.讨论下列方程组,当$\lambda$取什么值时有唯一解?取什么值时有无穷多解?取什么值时无解?\\
\noindent (1)
$$
\left\{
\begin{aligned}
(\lambda +3)x_{1}+x_{2}+2x_{3}=\lambda \\
\lambda x_{1}+(\lambda -1)x_{2} + x_{3}=\lambda \\
3(\lambda +1)x_{1}+\lambda x_{2}+(\lambda +3)x_{3} = 3
\end{aligned}
\right.
$$
\begin{proof}[解]
	$
	\left[
	\begin{matrix}
	\lambda + 3 & 1 & 2 & \lambda \\
	\lambda & \lambda - 1 & 1 & \lambda \\
	3(\lambda + 1) & \lambda & \lambda + 3 & 3
	\end{matrix}
	\right] \xrightarrow[R_{3}-3R_{2}]{R_{1}-R_{2}} \left[
	\begin{matrix}
	3 & 2-\lambda & 1 & 0 \\
	\lambda & \lambda - 1 & 1 & \lambda \\
	3 & 3 - 2\lambda & \lambda & 3 - 3\lambda
	\end{matrix}
	\right]\\
	\xrightarrow[R_{3}-R_{1}]{3R_{2}-\lambda R_{1}} \left[
	\begin{matrix}
	3 & 2-\lambda & 1 & 0 \\
	0 & \lambda^{2} + \lambda -3 & 3-\lambda & 3\lambda \\
	0 & 1 - \lambda & \lambda -1 & 3(1 - \lambda)
	\end{matrix}
	\right]
	$
	当$\lambda = 1$时,矩阵为\\
	$$
	\left[
	\begin{matrix}
	3 & 1 & 1 & 0 \\
	0 & -1 & 2 & 3 \\
	0 & 0 & 0 & 0
	\end{matrix}
	\right]
	$$
	此时,$r(A|b) = r(A) < 3$,所以有无穷多解.
	当$\lambda \neq 1$时,继续初等行变换\\
	$
	\xrightarrow{R_{3}/(1-\lambda)}
	\left[
	\begin{matrix}
	3 & 2-\lambda & 1 & 0 \\
	0 & \lambda^{2} + \lambda -3 & 3-\lambda & 3\lambda \\
	0 & 1 & -1 & 3
	\end{matrix}
	\right] \xrightarrow{R_{2} - (\lambda^{2} + \lambda -3)R_{3}}
	\left[
	\begin{matrix}
	3 & 2-\lambda & 1 & 0 \\
	0 & \lambda^{2} + \lambda -3 & 3-\lambda & 3\lambda \\
	0 & 1 & -1 & 3
	\end{matrix}
	\right]\\
	\xrightarrow{R_{23}}
	\left[
	\begin{matrix}
	3 & 2-\lambda & 1 & 0 \\
	0 & 1 & -1 & 3 \\
	0 & 0 & \lambda^{2} & 9-3{\lambda}^{2} 
	\end{matrix}
	\right]\\
	$
	当$r(A|b) > r(A)$时,无解,此时$\lambda = 0$.\\
	所以,当$\lambda \neq 1 $ 且 $ \lambda \neq 0$时,$r(A) = r(A|b) = 3$, 此时有唯一解.
\end{proof}
\noindent (2)
$$
\left\{
\begin{aligned}
x_{1}+2x_{2}+\lambda x_{3}=1 \\
2x_{1}+\lambda x_{2} + 8x_{3}=\lambda 
\end{aligned}
\right.
$$
\begin{proof}[解]
	$$
	\left[
	\begin{matrix}
	1 & 2 & \lambda & 1 \\
	2 & \lambda & 8 & \lambda 
	\end{matrix}
	\right] \xrightarrow{R_{2}-2R_{1}}
	\left[
	\begin{matrix}
	1 & 2 & \lambda & 1 \\
	0 & \lambda - 4 & 8 - 2\lambda & \lambda -2
	\end{matrix}
	\right]
	$$
	1). 当$\lambda = 4$时,$r(A) < r(A|b)$ ,则无解.\\
	2). 当$\lambda \neq 4$时, $r(A) = r(A|b) < 3$, 则有无穷多组解.
\end{proof}
\noindent (3)
$$
\left\{
\begin{aligned}
x_{1}+x_{2}+\lambda x_{3}=2 \\
3x_{1}+4x_{2} + 2x_{3}=\lambda \\
2x_{1}+3x_{2}-x_{3} = 3
\end{aligned}
\right.
$$
\begin{proof}[解]
	$$
	\left[
	\begin{matrix}
	1 & 1 & \lambda & 2 \\
	3 & 4 & 2 & \lambda \\
	2 & 3 & -1 & 3
	\end{matrix}
	\right] \xrightarrow[R_{3}-2R_{1}]{R_{2}-3R_{1}}
	\left[
	\begin{matrix}
	1 & 1 & \lambda & 2 \\
	0 & 1 & 2-3\lambda & \lambda - 6 \\
	0 & 1 & -1-2\lambda & -1
	\end{matrix}
	\right] \xrightarrow{R_{3}-R_{2}}
	\left[
	\begin{matrix}
	1 & 1 & \lambda & 2 \\
	0 & 1 & 2-3\lambda & \lambda - 6 \\
	0 & 0 & \lambda -3& 5-\lambda
	\end{matrix}
	\right]
	$$
	1). 当$\lambda = 3$时, $r(A) < r(A|b)$,则无解.\\
	2). 当$\lambda \neq 3$时, $r(A) = r(A|b) = 3$, 有唯一解.
\end{proof}
\noindent 5.讨论$a,b$为何值时,线性方程组
$$
\left\{
\begin{aligned}
x_{1}-x_{2}+2x_{3}=1\\
2x_{1}-x_{2}+3x_{3}-x_{4}=4\\
x_{2}+ax_{3}+bx_{4}=b\\
x_{1}-3x_{2}+(3-a)x_{3}=-4
\end{aligned}
\right.
$$
有解或无解,若有解,求出其解.
\begin{proof}[解]
	$$
	\left[
	\begin{matrix}
	1 & -1 & 2 & 0 & 1\\
	2 & -1 & 3 & -1 & 4\\
	0 & 1 & a & b & b\\
	1 & -3 & 3-a & 0 & -4
	\end{matrix}
	\right]
	$$
	$$
	\xrightarrow[R_{4}-R_{1}]{R_{2}-2R_{1}}
	\left[
	\begin{matrix}
	1 & -1 & 2 & 0 & 1\\
	0 & 1 & -1 & -1 & 2\\
	0 & 1 & a & b & b\\
	0 & -2 & 1-a & 0 & -5
	\end{matrix}
	\right]
	$$
	$$
	\xrightarrow[R_{3}-R_{2}]{R_{4}+2R_{2}}
	\left[
	\begin{matrix}
	1 & -1 & 2 & 0 & 1\\
	0 & 1 & -1 & -1 & 2\\
	0 & 0 & a+1 & b+1 & b-2\\
	0 & 0 & -(1+a) & -2 & -1
	\end{matrix}
	\right]
	$$
	$$
	\xrightarrow{R_{4}+R_{3}}
	\left[
	\begin{matrix}
	1 & -1 & 2 & 0 & 1\\
	0 & 1 & -1 & -1 & 2\\
	0 & 0 & a+1 & b+1 & b-2\\
	0 & 0 & 0 & b-1 & b-3
	\end{matrix}
	\right]
	$$
	1. 当$b=1$时,$r(A) < r(A|b)$,方程组无解.\\
	2. 当$b \neq 1, a = -1$时,
	$$
	\left[
	\begin{matrix}
	1 & -1 & 2 & 0 & 1\\
	0 & 1 & -1 & -1 & 2\\
	0 & 0 & 0 & b+1 & b-2\\
	0 & 0 & 0 & b-1 & b-3
	\end{matrix}
	\right]
	$$
	1)当$b=-1$时,$r(A) < r(A|b)$,方程组无解.\\
	2)当$b \neq -1$时,
	$$
	\left[
	\begin{matrix}
	1 & -1 & 2 & 0 & 1\\
	0 & 1 & -1 & -1 & 2\\
	0 & 0 & 0 & b+1 & b-2\\
	0 & 0 & 0 & b-1 & b-3
	\end{matrix}
	\right] 
	$$
	$$
	\xrightarrow{R_{3}/(b+1)}
	\left[
	\begin{matrix}
	1 & -1 & 2 & 0 & 1\\
	0 & 1 & -1 & -1 & 2\\
	0 & 0 & 0 & 1 & (b-2)/(b+1)\\
	0 & 0 & 0 & b-1 & b-3
	\end{matrix}
	\right]\\
	$$
	$$
	\xrightarrow{R_{4}-(b-1)R_{3}}
	\left[
	\begin{matrix}
	1 & -1 & 2 & 0 & 1\\
	0 & 1 & -1 & -1 & 2\\
	0 & 0 & 0 & 1 & (b-2)/(b+1)\\
	0 & 0 & 0 & 0 & (b-5)/(b+1)
	\end{matrix}
	\right]
	$$
	i)当$b=5$时
	$$
	\left[
	\begin{matrix}
	1 & -1 & 2 & 0 & 1\\
	0 & 1 & -1 & -1 & 2\\
	0 & 0 & 0 & 1 & 1/2\\
	0 & 0 & 0 & 0 & 0
	\end{matrix}
	\right]
	$$
	$$
	\xrightarrow{R_{2}+R_{3}}
	\left[
	\begin{matrix}
	1 & -1 & 2 & 0 & 1\\
	0 & 1 & -1 & 0 & 5/2\\
	0 & 0 & 0 & 1 & 1/2\\
	0 & 0 & 0 & 0 & 0
	\end{matrix}
	\right]
	$$
	$$
	\xrightarrow{R_{1}+R_{2}}
	\left[
	\begin{matrix}
	1 & 0 & 1 & 0 & 7/2\\
	0 & 1 & -1 & 0 & 5/2\\
	0 & 0 & 0 & 1 & 1/2\\
	0 & 0 & 0 & 0 & 0
	\end{matrix}
	\right]
	$$
	所以,$x_{1},x_{2},x_{4}$为主元变量,$x_{3}$为自由变量,解得
	$$
	\left\{
	\begin{aligned}
	x_{1} = & 3.5-t_{3} \\
	x_{2} = & 2.5+t_{3} \\
	x_{3} =  & t_{3}\\
	x_{4} = & 0.5
	\end{aligned}
	\right.
	$$
	ii)当$b \neq 5$时,$r(A) < r(A|b)$,方程组无解.\\
	3.当$b \neq 1, a \neq -1$时,$r(A) = r(A|b) = 4$,所以有唯一解
	$$
	\left\{
	\begin{aligned}
	x_{1} = & \frac{4ab+5b-6a-11}{(b-1)(a+1)} \\
	x_{2} = & \frac{3ab+2b-5a}{(b-1)(a+1)} \\
	x_{3} =  & \frac{5-b}{(b-1)(a+1)}\\
	x_{4} = & \frac{b-3}{b-1}
	\end{aligned}
	\right.
	$$
\end{proof}
\noindent 7.已知平面上三条不同直线方程分别为
$$
l_{1}:ax+2by+3c=0
$$
$$
l_{2}:bx+2cy+3a=0
$$
$$
l_{3}:cx+2ay+3b=0
$$
试证这三条直线交于一点的充要条件是$a+b+c=0$.
\begin{proof}[解]
	考虑方程组
	$$
	\left\{
	\begin{aligned}
	ax+2by=-3c \\
	bx+2cy=-3a \\
	cx+2ay=-3b
	\end{aligned}
	\right.
	$$
	其增广矩阵为
	$$
	\left[
	\begin{matrix}
	a & 2b & -3c \\
	b & 2c & -3a \\
	c & 2a & -3b 
	\end{matrix}
	\right]
	$$
	三条直线交于一点,当且仅当方程组有唯一解,当且仅当$r(A)=r(A|b)=2$\\
	$
	\left[
	\begin{matrix}
	a & 2b & -3c \\
	b & 2c & -3a \\
	c & 2a & -3b 
	\end{matrix}
	\right] \xrightarrow{R_{1}+R_{2}+R_{3}}
	\left[
	\begin{matrix}
	a+b+c & 2(a+b+c) & -3(a+b+c) \\
	b & 2c & -3a \\
	c & 2a & -3b 
	\end{matrix}
	\right]
	$\\
	1. 当$a+b+c \neq 0$时, 继续做初等行变换
	$$
	\xrightarrow{R_{1}/(a+b+c)} 
	\left[
	\begin{matrix}
	1 & 2 & -3 \\
	b & 2c & -3a \\
	c & 2a & -3b 
	\end{matrix}
	\right]
	$$
	$$
	\xrightarrow[R_{3}-cR_{1}]{R_{2}-bR_{1}}
	\left[
	\begin{matrix}
	1 & 2 & -3 \\
	0 & 2(c-b) & 3(b-a) \\
	0 & 2(a-c) & 3(c-b)
	\end{matrix}
	\right]
	$$
	$$
	\xrightarrow{(c-b)R_{3}-(a-c)R_{2}}
	\left[
	\begin{matrix}
	1 & 2 & -3 \\
	0 & 2(c-b) & 3(b-a) \\
	0 & 0 & 3((c-b)^{2}-(b-a)(a-c))
	\end{matrix}
	\right]
	$$
	注意$3((c-b)^{2}-(b-a)(a-c)) = 3(a^{2}+b^{2}+c^{2}-ab-bc-ac)\\
	=\frac{3}{2}((a-b)^{2}+(b-c)^{2}+(c-a)^{2}) > 0$, 因为$a-b,b-c,c-a$不能同时为零,否则$a=b=c$,三条直线相同.\\
	所以,我们有$r(A) < r(A|b)$,方程组无解.\\
	2.当$a+b+c = 0$时,矩阵为
	$$
	\left[
	\begin{matrix}
	0 & 0 & 0 \\
	b & 2c & -3a \\
	c & 2a & -3b 
	\end{matrix}
	\right]
	$$
	$$
	\xrightarrow{bR_{3}-cR_{2}}
	\left[
	\begin{matrix}
	0 & 0 & 0 \\
	b & 2c & -3a \\
	0 & 2(ab-c^{2}) & -3(b^{2}-ac)
	\end{matrix}
	\right]
	$$
	注意到$ab-c^{2} = ab-(a+b)^{2} = -(a^{2}+ab+b^{2}) = -(a+\frac{1}{2}b)^{2}-\frac{3}{4}b^{2} < 0$, 因为$a,b$不能同时为0,否则$a,b,c$都为0.\\
	所以,我们有$r(A) = r(A|b) = 2$,方程组有唯一解.\\
	综上所述,方程组有唯一解的充要条件是$a+b+c = 0$.
\end{proof}
\noindent 16.设
$$
A=\left[
\begin{matrix}
1 & -1 & -1 \\
-1 & 1 & 1 \\
0 & -4 & -2
\end{matrix}
\right],\xi_{1} = \left[
\begin{matrix}
-1 \\
1 \\
-2
\end{matrix}
\right]
$$
(1)求满足$A\xi_{2}=\xi_{1},A^{2}\xi_{3}=\xi_{1}$的所有向量$\xi_{2},\xi_{3}$.\\
(2)对(1)中的任意向量$\xi_{2},\xi_{3}$证明$\xi_{1},\xi_{2},\xi_{3}$线性无关.
\begin{proof}[解]
	(1)$[A|\xi_{1}] = \left[
	\begin{matrix}
	1 & -1 & -1 & -1\\
	-1 & 1 & 1 & 1\\
	0 & -4 & -2 & -2
	\end{matrix}
	\right] \xrightarrow[-R_{3}/4]{R_{2}+R_{1}} \left[
	\begin{matrix}
	1 & -1 & -1 & -1\\
	0 & 0 & 0 & 0\\
	0 & 1 & 1/2 & 1/2
	\end{matrix}
	\right] \xrightarrow[R_{23}]{R_{1}+R_{2}} \left[
	\begin{matrix}
	1 & 0 & -1/2 & -1/2\\
	0 & 1 & 1/2 & 1/2\\
	0 & 0 & 0 & 0
	\end{matrix}
	\right]
	$
	所以,$\xi_{2} = \left[
	\begin{matrix}
	0 \\
	0 \\
	1
	\end{matrix}
	\right]+t\left[
	\begin{matrix}
	1 \\
	-1 \\
	2
	\end{matrix}
	\right]$($t$为任意实数)
	$\\
	A^{2}= \left[
	\begin{matrix}
	1 & -1 & -1\\
	-1 & 1 & 1\\
	0 & -4 & -2
	\end{matrix}
	\right] \left[
	\begin{matrix}
	1 & -1 & -1\\
	-1 & 1 & 1\\
	0 & -4 & -2
	\end{matrix}
	\right] = \left[
	\begin{matrix}
	1 & -1 & -1\\
	-1 & 1 & 1\\
	0 & -4 & -2
	\end{matrix}
	\right] = \left[
	\begin{matrix}
	2 & 2 & 0\\
	-2 & -2 & 0\\
	4 & 4 & 0
	\end{matrix}
	\right]
	$
	$
	[A^{2}|\xi_{1}]=\left[
	\begin{matrix}
	2 & 2 & 0 & -1\\
	-2 & -2 & 0 & 1\\
	4 & 4 & 0 & -2
	\end{matrix}
	\right] \xrightarrow[R_{4}-2R_{2}]{R_{3}+R_{1}} \left[
	\begin{matrix}
	2 & 2 & 0 & -1\\
	0 & 0 & 0 & 0\\
	0 & 0 & 0 & 0
	\end{matrix}
	\right]	
	$
	得到基础解系
	$
	\left[
	\begin{matrix}
	1 \\
	-1 \\
	0
	\end{matrix}
	\right], \left[
	\begin{matrix}
	0 \\
	0 \\
	1
	\end{matrix}
	\right]
	$, 特解
	$
	\left[
	\begin{matrix}
	-1/2 \\
	0 \\
	0
	\end{matrix}
	\right]
	$\\
	所以$ \xi_{3} = \left[
	\begin{matrix}
	-1/2 \\
	0 \\
	0
	\end{matrix}
	\right] + t_{1}\left[
	\begin{matrix}
	1 \\
	-1 \\
	0
	\end{matrix}
	\right]+t_{2}\left[
	\begin{matrix}
	0 \\
	0 \\
	1
	\end{matrix}
	\right]
	$($t_{1}, t_{2}$为任意实数)\\
	\\
	(2)假设$\xi_{1},\xi_{2},\xi_{3}$线性相关,则存在不全为0的数使得
	$$k_{1}\xi_{1}+k_{2}\xi_{2}+k_{3}\xi_{3} = 0$$
	$$k_{1}\left[
	\begin{matrix}
	-1 \\
	1 \\
	-2
	\end{matrix}
	\right]+k_{2}\left[
	\begin{matrix}
	t \\
	-t \\
	2t+1
	\end{matrix}
	\right]+k_{3}\left[
	\begin{matrix}
	t_{1}-1/2 \\
	-t_{1} \\
	t_{2}
	\end{matrix}
	\right] = 0$$
	系数矩阵为
	$$
	\left[
	\begin{matrix}
	-1 & t & t_{1}-1/2 \\
	1 & -t & -t_{1} \\
	-2 & 2t+1 & t_{2}
	\end{matrix}
	\right]
	$$
	$$
	\xrightarrow[R_{3}-2R_{1}]{R_{2}+R_{1}}
	\left[
	\begin{matrix}
	-1 & t & t_{1}-1/2 \\
	0 & 0 & -1/2 \\
	0 & 1 & t_{2}-2t_{1}+1
	\end{matrix}
	\right] \xrightarrow{R_{23}}
	\left[
	\begin{matrix}
	-1 & t & t_{1}-1/2 \\
	0 & 1 & t_{2}-2t_{1}+1\\
	0 & 0 & -1/2 
	\end{matrix}
	\right]
	$$
	所以,$r(A) = 3$, 方程组只有零解,$k_{1},k_{2},k_{3}$全为0, $\xi_{1},\xi_{2},\xi_{3}$线性无关.
\end{proof}
\noindent 17.设向量组$\alpha_{1} = [1,0,1]^{T}, \alpha_{2} = [0,1,1]^{T}, \alpha_{3} = [1,3,5]^{T}$不能由向量组$\beta_{1}=[1,1,1]^{T},\beta_{2}=[1,2,3]^{T},\beta_{3}=[3,4,a]^{T}$线性表出.
\begin{proof}[解]
	因为向量组$\alpha_{1}, \alpha_{2}, \alpha_{3}$不能由$\beta_{1}, \beta_{2}, \beta_{3}$线性表出,所以$r(\beta_{1}, \beta_{2}, \beta_{3}) < r(\beta_{1}, \beta_{2}, \beta_{3}, \alpha_{1}, \alpha_{2}, \alpha_{3})$\\
	$$
	[\beta_{1}, \beta_{2}, \beta_{3}, \alpha_{1}, \alpha_{2}, \alpha_{3}] = \left[
	\begin{matrix}
	1 & 1 & 3 & 1 & 0 & 1\\
	1 & 2 & 4 & 0 & 1 & 3\\
	1 & 3 & a & 1 & 1 & 5
	\end{matrix}
	\right]
	$$
	$$
	\xrightarrow[R_{3}-R_{1}]{R_{2}-R_{1}}
	\left[
	\begin{matrix}
	1 & 1 & 3 & 1 & 0 & 1\\
	0 & 1 & 1 & -1 & 1 & 2\\
	0 & 2 & a-3 & 0 & 1 & 4
	\end{matrix}
	\right]
	$$
	$$
	\xrightarrow{R_{3}-2R_{2}}
	\left[
	\begin{matrix}
	1 & 1 & 3 & 1 & 0 & 1\\
	0 & 1 & 1 & -1 & 1 & 2\\
	0 & 0 & a-5 & 2 & -1 & 0
	\end{matrix}
	\right]
	$$
	所以当$a = 5$时,$r(\beta_{1}, \beta_{2}, \beta_{3}) < r(\beta_{1}, \beta_{2}, \beta_{3}, \alpha_{1}, \alpha_{2}, \alpha_{3})$
	$$
	[\alpha_{1}, \alpha_{2}, \alpha_{3}, \beta_{1}, \beta_{2}, \beta_{3}] = \left[
	\begin{matrix}
	1 & 0 & 1 & 1 & 1 & 3 \\
	0 & 1 & 3 & 1 & 2 & 4 \\
	1 & 1 & 5 & 1 & 3 & 5 &
	\end{matrix}
	\right]
	$$
	$$
	\xrightarrow{R_{3}-R_{1}} 
	\left[
	\begin{matrix}
	1 & 0 & 1 & 1 & 1 & 3 \\
	0 & 1 & 3 & 1 & 2 & 4 \\
	0 & 1 & 4 & 0 & 2 & 2 &
	\end{matrix}
	\right]
	$$
	$$
	\xrightarrow{R_{3}-R_{2}} 
	\left[
	\begin{matrix}
	1 & 0 & 1 & 1 & 1 & 3 \\
	0 & 1 & 3 & 1 & 2 & 4 \\
	0 & 0 & 1 & -1 & 0 & -2 &
	\end{matrix}
	\right]
	$$
	$$
	\xrightarrow[R_{1}-R_{3}]{R_{2}-3R_{3}} 
	\left[
	\begin{matrix}
	1 & 0 & 0 & 2 & 1 & 5 \\
	0 & 1 & 0 & 4 & 2 & 10 \\
	0 & 0 & 1 & -1 & 0 & -2 &
	\end{matrix}
	\right]
	$$
	所以,我们得到
	$$
	\left\{
	\begin{aligned}
	\beta_{1} = & 2\alpha_{1}+4\alpha_{2}-\alpha_{3} \\
	\beta_{2} = & \alpha_{1}+2\alpha_{2} \\
	\beta_{3} = & 5\alpha_{1}+10\alpha_{2}-2\alpha_{3} 
	\end{aligned}
	\right.
	$$
\end{proof}
\noindent 18.设矩阵
$$
\left[
\begin{matrix}
1 & -2 & 3 & -4 \\
0 & 1 & -1 & 1 \\
1 & 2 & 0 & -3
\end{matrix}
\right]
$$
(1)求方程组$Ax=0$的一个基础解系;\\
(2)求满足$AB=I$的所有矩阵$B$,其中$I$是3阶单位矩阵.
\begin{proof}[解]
	(1)
	$
	\left[
	\begin{matrix}
	1 & -2 & 3 & -4 \\
	0 & 1 & -1 & 1 \\
	1 & 2 & 0 & -3
	\end{matrix}
	\right] \xrightarrow{R_{3}-R_{2}}
	\left[
	\begin{matrix}
	1 & -2 & 3 & -4 \\
	0 & 1 & -1 & 1 \\
	0 & 4 & -3 & 1
	\end{matrix}
	\right] \xrightarrow{R_{3}-4R_{2}}
	\left[
	\begin{matrix}
	1 & -2 & 3 & -4 \\
	0 & 1 & -1 & 1 \\
	0 & 0 & 1 & -3
	\end{matrix}
	\right] \xrightarrow{R_{2}+R_{3}}
	\left[
	\begin{matrix}
	1 & -2 & 3 & -4 \\
	0 & 1 & 0 & -2  \\
	0 & 0 & 1 & -3
	\end{matrix}
	\right] \xrightarrow{R_{1}+2R_{2}-3R_{1}}
	\left[
	\begin{matrix}
	1 & 0 & 0 & 1 \\
	0 & 1 & 0 & -2  \\
	0 & 0 & 1 & -3
	\end{matrix}
	\right]
	$\\
	所以,基础解系为$(-1,2,3,1)^{T}$.\\
	(2)$A$为$3 \times 4$矩阵, $I$为3阶单位矩阵,所以$B$为 $4 \times 3$ 矩阵, 设$B$的列向量为$\alpha_{1}, \alpha_{2}, \alpha_{3}$. 即求$A(\alpha_{1}, \alpha_{2}, \alpha_{3}) = I$的所有解,即求$A\alpha_{1} =\left[
	\begin{matrix}
	1\\
	0\\
	0 
	\end{matrix}
	\right] , A\alpha_{2} = \left[
	\begin{matrix}
	0\\
	1\\
	0 
	\end{matrix}
	\right], A\alpha_{3} = \left[
	\begin{matrix}
	0\\
	0\\
	1 
	\end{matrix}
	\right]$的所有解.\\
	$[A|I] = \left[
	\begin{matrix}
	1 & -2 & 3 & -4 & 1 & 0 & 0\\
	0 & 1 & -1 & 1 & 0 & 1 & 0\\
	1 & 2 & 0 & -3 & 0 & 0 & 1
	\end{matrix}
	\right] \xrightarrow{R_{3}-R_{2}} 
	 \left[
	\begin{matrix}
	1 & -2 & 3 & -4 & 1 & 0 & 0\\
	0 & 1 & -1 & 1 & 0 & 1 & 0\\
	0 & 4 & -3 & 1 & -1 & 0 & 1
	\end{matrix}
	\right] \xrightarrow{R_{3}-4R_{2}}
	\left[
	\begin{matrix}
	1 & -2 & 3 & -4 & 1 & 0 & 0\\
	0 & 1 & -1 & 1 & 0 & 1 & 0\\
	0 & 0 & 1 & -3 & -1 & -4 & 1
	\end{matrix}
	\right] \xrightarrow{R_{2}+R_{3}}
	\left[
	\begin{matrix}
	1 & -2 & 3 & -4 & 1 & 0 & 0\\
	0 & 1 & 0 & -2 & -1 & -3 & 1\\
	0 & 0 & 1 & -3 & -1 & -4 & 1
	\end{matrix}
	\right] \xrightarrow{R_{1}+2R_{2}-3R_{3}}
	\left[
	\begin{matrix}
	1 & 0 & 0 & 1 & 2 & 6 & -1\\
	0 & 1 & 0 & -2 & -1 & -3 & 1\\
	0 & 0 & 1 & -3 & -1 & -4 & 1
	\end{matrix}
	\right]
	$ \\
	解得$\alpha_{1} = k_{1}(-1,2,3,1)^{T}+(2,-1,-1,0)^{T}, \alpha_{2} = k_{2}(-1,2,3,1)^{T}+(6,-3,-4,0)^{T}, \alpha_{3} = k_{3}(-1,2,3,1)^{T}+(-1,1,1,0)^{T}$\\
	所以,把$\alpha_{1},\alpha_{2},\alpha_{3}$代入$B=(\alpha_{1},\alpha_{2},\alpha_{3})$,即得所有的$B$.
\end{proof}
\noindent 21.设$A$是$s \times n$矩阵, $B$是$n \times m$矩阵, $n < m$,证明:齐次线性方程组$(AB)x=0$有非零解.
\begin{proof}[证]
	$r(B) \leq min\{m, n\} = n < m$, 所以$Bx =0$,有非零解.不妨记为$x_{1}$.\\
	所以$(AB)x_{1}=A(Bx_{1})=0$.\\
	即$x_{1}$也是$(AB)x=0$的非零解.\\
	所以,$(AB)x=0$有非零解.
\end{proof}
\noindent 22.设$A$是$m \times s$矩阵,$B$是$s \times n$矩阵,$x$是$n$元向量.证明:若$(AB)x=0$与$Bx=0$是同解方程组,则$rank(AB)=rank(A)$.
\begin{proof}[证]
	$(AB)x=0$与$Bx=0$是同解方程组,所以,两个方程组里基础解系中向量个数相等.\\
	$AB$为$m \times n$的矩阵, $B$是$s \times n$矩阵.\\
	所以$n-r(AB) = n-r(B)$, $r(AB)=r(B)$.
\end{proof}
\noindent 24.问$y$为何值时,向量$\beta = [4,6,y,-2]^{T}$可以由向量组$\alpha_{1} = [2,3,1,-2]^{T}, \alpha_{2} = [3,-2,-5,3]^{T},\alpha_{3} = [-3,2,2,-1]^{T}$线性表出.
\begin{proof}[证]
	$[\alpha_{1},\alpha_{2},\alpha_{3},\beta] = \left[
	\begin{matrix}
	2 & 3 & -3 & 4 \\
	3 & -2 & 2 & 6  \\
	1 & -5 & 2 & y\\
	-2 & 3 & -1 & -2
	\end{matrix}
	\right] \xrightarrow[2R_{3}-R_{1};R_{4}+R_{1}]{2R_{2}-3R_{1}} \left[
	\begin{matrix}
	2 & 3 & -3 & 4 \\
	0 & 1 & -1 & 0  \\
	0 & -13 & 7 & 2y-4\\
	0 & 3 & -2 & 1
	\end{matrix} 
	\right] \xrightarrow[R_{4}-3R_{2}]{R_{3}+13R_{2}} \left[
	\begin{matrix}
	2 & 3 & -3 & 4 \\
	0 & 1 & -1 & 0  \\
	0 & 0 & 3 & 2-y\\
	0 & 0 & 1 & 1
	\end{matrix} 
	\right] \xrightarrow{3R_{4}-R_{3}} \left[
	\begin{matrix}
	2 & 3 & -3 & 4 \\
	0 & 1 & -1 & 0  \\
	0 & 0 & 3 & 2-y\\
	0 & 0 & 0 & 1+y
	\end{matrix} 
	\right]$\\
	所以,当$1+y=0,y=-1$时,向量$\beta$可以由向量组$\alpha_{1}, \alpha_{2},\alpha_{3}$线性表出.
\end{proof}
\noindent 26.求下列向量组的秩和极大线性无关组:\\
\noindent (1)$\alpha_{1}=[2,1,0]^{T}, \alpha_{2}=[3,1,1]^{T}, \alpha_{3}=[2,0,2]^{T}, \alpha_{4}=[4,2,0]^{T}$;
\begin{proof}[解]
	$[\alpha_{1}, \alpha_{2}, \alpha_{3}, \alpha_{4}] = \left[
	\begin{matrix}
	2 & 3 & 2 & 4 \\
	1 & 1 & 0 & 2 \\
	0 & 1 & 2 & 0 
	\end{matrix}
	\right] \xrightarrow{2R_{2}-R_{1}} \left[
	\begin{matrix}
	2 & 3 & 2 & 4 \\
	0 & -1 & -2 & 0 \\
	0 & 1 & 2 & 0 
	\end{matrix}
	\right] \\
	\xrightarrow{R_{3}+R_{2}} 
	\left[
	\begin{matrix}
	2 & 3 & 2 & 4 \\
	0 & -1 & -2 & 0 \\
	0 & 0 & 0 & 0 
	\end{matrix}
	\right]$\\
	所以,向量组的秩为2,极大线性无关组可以为$\alpha_{1}, \alpha_{2};\alpha_{1}, \alpha_{3};\alpha_{2}, \alpha_{3};\alpha_{2}, \alpha_{4};\alpha_{3}, \alpha_{4}$
\end{proof}
\noindent (2)$\alpha_{1}=[1,1,1,1]^{T}, \alpha_{2}=[1,1,-1,-1]^{T}, \alpha_{3}=[1,-1,-1,1]^{T}, \alpha_{4}=[-1,-1,-1,1]^{T}$;
\begin{proof}[解]
	$[\alpha_{1}, \alpha_{2}, \alpha_{3}, \alpha_{4}] = \left[
	\begin{matrix}
	1 & 1 & 1 & -1 \\
	1 & 1 & -1 & -1 \\
	1 & -1 & -1 & -1 \\
	1 & -1 & 1 & 1 
	\end{matrix}
	\right] \xrightarrow[R_{3}-R_{1};R_{4}-R_{1}]{R_{2}-R_{1}} \left[
	\begin{matrix}
	1 & 1 & 1 & -1 \\
	0 & 0 & -2 & 0 \\
	0 & -2 & -2 & 0 \\
	0 & -2 & 0 & 2 
	\end{matrix}
	\right] \\
	\xrightarrow[R_{24}]{-1/2R_{2};-1/2R_{3};-1/2R_{4}} 
	\left[
	\begin{matrix}
	1 & 1 & 1 & -1 \\
	0 & 1 & 0 & -1 \\
	0 & 1 & 1 & 0 \\
	0 & 0 & 1 & 0
	\end{matrix}
	\right] 
	\xrightarrow{R_{3}-R_{2}}
	\left[
	\begin{matrix}
	1 & 1 & 1 & -1 \\
	0 & 1 & 0 & -1 \\
	0 & 0 & 1 & 1 \\
	0 & 0 & 1 & 0
	\end{matrix}
	\right] \\
	\xrightarrow{R_{4}-R_{3}}
	\left[
	\begin{matrix}
	1 & 1 & 1 & -1 \\
	0 & 1 & 0 & -1 \\
	0 & 0 & 1 & 1 \\
	0 & 0 & 0 & -1
	\end{matrix}
	\right] \\
	$\\
	所以,向量组的秩为4,极大线性无关组可以为$\alpha_{1}, \alpha_{2}, \alpha_{3}, \alpha_{4}$
\end{proof}
\noindent (3)$\alpha_{1}=[6,4,1,-1,2]^{T}, \alpha_{2}=[1,0,2,3,-4]^{T}, \alpha_{3}=[1,4,-9,-16,22]^{T}, \alpha_{4}=[7,1,0,-1,3]^{T}$;
\begin{proof}[解]
	$$
	[\alpha_{1}, \alpha_{2}, \alpha_{3}, \alpha_{4}] = \left[
	\begin{matrix}
	6 & 1 & 1 & 7 \\
	4 & 0 & 4 & 1 \\
	1 & 2 & -9 & 0 \\
	-1 & 3 & -16 & -1\\
	2 & -4 & 22 & 3 
	\end{matrix}
	\right]	
	$$
	$$
	\xrightarrow{R_{1}-6R_{3};R_{2}-4R_{3};R_{4}+R_{3};R_{5}-2R_{3}}
	\left[
	\begin{matrix}
	0 & -11 & 55 & 7 \\
	0 & -8 & 40 & 1 \\
	1 & 2 & -9 & 0 \\
	0 & 5 & -25 & -1\\
	0 & -8 & 40 & 3 
	\end{matrix}
	\right]
	$$
	$$
	\xrightarrow{...}
	\left[
	\begin{matrix}
	1 & 2 & -9 & 0 \\
	0 & 1 & -5 & -1/5 \\
	0 & 0 & 0 & 1 \\
	0 & 0 & 0 & 0\\
	0 & 0 & 0 & 0 
	\end{matrix}
	\right]
	$$
	$$
	\xrightarrow{...}
	\left[
	\begin{matrix}
	1 & 0 & 1 & 0 \\
	0 & 1 & -5 & 0 \\
	0 & 0 & 0 & 1 \\
	0 & 0 & 0 & 0\\
	0 & 0 & 0 & 0 
	\end{matrix}
	\right]
	$$
	所以,向量组的秩为3,极大线性无关组可以为$\alpha_{1}, \alpha_{2}, \alpha_{4};\alpha_{1}, \alpha_{3}, \alpha_{4};\alpha_{2}, \alpha_{3}, \alpha_{4}$
\end{proof}
\noindent 27.已知向量组$\alpha_{1}=[1,-1,2,1,0]^{T}, \alpha_{2}=[2,-2,4,-2,0]^{T}, \alpha_{3}=[3,0,6,-1,1]^{T}, \alpha_{4}=[0,x,0,0,1]^{T}$有4个不同的极大线性无关组求$x$的值.
\begin{proof}[解]
	$$
	[\alpha_{1}, \alpha_{2}, \alpha_{3}, \alpha_{4}] = \left[
	\begin{matrix}
	1 & 2 & 3 & 0 \\
	-1 & -2 & 0 & x \\
	2 & 4 & 6 & 0 \\
	1 & -2 & -1 & 0\\
	0 & 0 & 1 & 1
	\end{matrix}
	\right]	
	$$
	$$
	\xrightarrow{R_{2}+R_{1};R_{3}-2R_{1};R_{4}-R_{1}}
	 \left[
	\begin{matrix}
	1 & 2 & 3 & 0 \\
	0 & 0 & 3 & x \\
	0 & 0 & 0 & 0 \\
	0 & -4 & -4 & 0\\
	0 & 0 & 1 & 1
	\end{matrix}
	\right]
	$$
	$$
	\xrightarrow{-1/4R_{4};R_{2}-3R_{5}}
	\left[
	\begin{matrix}
	1 & 2 & 3 & 0 \\
	0 & 0 & 0 & x-3 \\
	0 & 0 & 0 & 0 \\
	0 & 1 & 1 & 0\\
	0 & 0 & 1 & 1
	\end{matrix}
	\right]
	$$
	$$
	\xrightarrow{}
	\left[
	\begin{matrix}
	1 & 2 & 3 & 0 \\
	0 & 1 & 1 & 0\\
	0 & 0 & 1 & 1\\
	0 & 0 & 0 & x-3 \\
	0 & 0 & 0 & 0 
	\end{matrix}
	\right]
	$$
	$$
	\xrightarrow{}
	\left[
	\begin{matrix}
	1 & 0 & 0 & -1 \\
	0 & 1 & 0 & -1\\
	0 & 0 & 1 & 1\\
	0 & 0 & 0 & x-3 \\
	0 & 0 & 0 & 0 
	\end{matrix}
	\right]
	$$
	1)$x-3 \neq 0, x \neq 3$时,只有一个极大线性无关组$\alpha_{1}, \alpha_{2}, \alpha_{3}, \alpha_{4}$.\\
	2)$x-3=0, x=3$时,有4个极大线性无关组,$\alpha_{1}, \alpha_{2}, \alpha_{3};\alpha_{1}, \alpha_{2}, \alpha_{4};\alpha_{1}, \alpha_{3}, \alpha_{4};\alpha_{2}, \alpha_{3}, \alpha_{4}$.\\
\end{proof}
\noindent 28.求下列齐次方程组的一个基础解系:\\
\noindent(1)
$$
\left\{
\begin{aligned}
x_{1}-2x_{2}+4x_{3}-7x_{4}=0 \\
2x_{1}+x_{2}-2x_{3}+x_{4}=0 \\
3x_{1}-x_{2}+2x_{3}-4x_{4}=0
\end{aligned}
\right.
$$
\begin{proof}[解]
	$
	\left[
	\begin{matrix}
	1 & -2 & 4 & -7 \\
	2 & 1 & -2 & 1\\
	3 & -1 & 2 & -4
	\end{matrix}
	\right]
	\xrightarrow{R_{2}-2R_{1};R_{3}-3R_{1}}
	\left[
	\begin{matrix}
	1 & -2 & 4 & -7 \\
	0 & 1 & -2 & 3\\
	0 & -5 & 10 & -17
	\end{matrix}
	\right]
	\xrightarrow{R_{3}+5R_{2}}
	\left[
	\begin{matrix}
	1 & -2 & 4 & -7 \\
	0 & 1 & -2 & 3\\
	0 & 0 & 0 & -2
	\end{matrix}
	\right]\\
	\xrightarrow{...}
	\left[
	\begin{matrix}
	1 & 0 & 0 & 0 \\
	0 & 1 & -2 & 0\\
	0 & 0 & 0 & 1
	\end{matrix}
	\right]
	$\\
	$n-r(A) = 4-3=1$,解得一个基础$(0,2,1,0)^{T}$
\end{proof}
\noindent (2)
$$
\left\{
\begin{aligned}
2x_{1}+x_{2}-x_{3}-x_{4}+x_{5}=0 \\
x_{1}-x_{2}+x_{3}+x_{4}-2x_{5}=0 \\
3x_{1}+3x_{2}-3x_{3}-3x_{4}+4x_{5}=0 \\
4x_{1}+5x_{2}-5x_{3}-5x_{4}+7x_{5}=0
\end{aligned}
\right.
$$
\begin{proof}[解]
	$
	\left[
	\begin{matrix}
	2 & 1 & -1 & -1 & 1 \\
	1 & -1 & 1 & 1 & -2\\
	3 & 3 & -3 & -3 & 4\\
	4 & 5 & -5 & -5 & 7
	\end{matrix}
	\right] 
	\xrightarrow{R_{1}-2R_{2};R_{3}-2R_{2};R_{4}-4R_{2}}
	\left[
	\begin{matrix}
	0 & -3 & 3 & 3 & -5 \\
	1 & -1 & 1 & 1 & -2\\
	0 & -3 & 3 & 3 & -5\\
	0 & -3 & 3 & 3 & -5
	\end{matrix}
	\right] \\
	\xrightarrow{3R_{2}-R_{1};R_{3}-R_{1};R_{4}-R_{1}}
	\left[
	\begin{matrix}
	0 & -3 & 3 & 3 & -5 \\
	3 & 0 & 0 & 0 & -1\\
	0 & 0 & 0 & 0 & 0\\
	0 & 0 & 0 & 0 & 0
	\end{matrix}
	\right]
	$\\
	$n-r(A)=5-2=3$,得到一个基础解系为$(0,1,1,0,0)^{T},(0,1,0,1,0)^{T},(\frac{1}{3},-\frac{5}{3},0,0,1)^{T}$
\end{proof}
\noindent 29.求下列线性方程组的通解,并表示成列向量线性组合的形式:\\
\noindent (1)
$$
\left\{
\begin{aligned}
x_{1}-2x_{2}-x_{3}-x_{4}+x_{5}=0 \\
2x_{1}+x_{2}-x_{3}+2x_{4}-3x_{5}=0 \\
3x_{1}-2x_{2}-x_{3}+x_{4}-2x_{5}=0 \\
2x_{1}-5x_{2}+x_{3}-2x_{4}+2x_{5}=0
\end{aligned}
\right.
$$
\begin{proof}[解]
	$
	\left[
	\begin{matrix}
	1 & -2 & -1 & -1 & 1 \\
	2 & 1 & -1 & 2 & -3\\
	3 & -2 & -1 & 1 & -2\\
	2 & -5 & 1 & -2 & 2
	\end{matrix}
	\right] 
	\xrightarrow{R_{2}-2R_{1};R_{3}-3R_{1};R_{4}-2R_{1}}
	\left[
	\begin{matrix}
	1 & -2 & -1 & -1 & 1 \\
	0 & -5 & -1 & -4 & 5 \\
	0 & -4 & -2 & -4 & 5\\
	0 & 1 & -3 & 0 & 0
	\end{matrix}
	\right]\\
	\xrightarrow{R_{2}+5R_{4};R_{3}+4R_{4}}
	\left[
	\begin{matrix}
	1 & -2 & -1 & -1 & 1 \\
	0 & 0 & -16 & -4 & 5 \\
	0 & 0 & -14 & -4 & 5\\
	0 & 1 & -3 & 0 & 0
	\end{matrix}
	\right]
	\xrightarrow{8R_{3}-7R_{2}}
	\left[
	\begin{matrix}
	1 & -2 & -1 & -1 & 1 \\
	0 & 0 & -16 & -4 & 5 \\
	0 & 0 & 0 & -4 & 5\\
	0 & 1 & -3 & 0 & 0
	\end{matrix}
	\right]\\
	\xrightarrow{}
	\left[
	\begin{matrix}
	1 & -2 & -1 & -1 & 1 \\
	0 & 1 & -3 & 0 & 0\\
	0 & 0 & -16 & -4 & 5 \\
	0 & 0 & 0 & -4 & 5
	\end{matrix}
	\right]\\
	$
	$n-r(A)=5-4=1$,得到一个基础解系为$(1,0,0,5,4)^{T}$,通解为
	$$
	\left[
	\begin{matrix}
	x_{1}\\
	x_{2}\\
	x_{3}\\
	x_{4}\\
	x_{5}
	\end{matrix}
	\right] = k
	\left[
	\begin{matrix}
	1\\
	0\\
	0\\
	5\\
	4
	\end{matrix}
	\right]
	$$
\end{proof}
\noindent (2)
$$
\left\{
\begin{aligned}
x_{1}+x_{2}+x_{3}+x_{4}+x_{5}=1 \\
3x_{1}+2x_{2}+x_{3}+x_{4}-3x_{5}=0 \\
1x_{2}+2x_{3}+2x_{4}+6x_{5}=3 \\
5x_{1}+4x_{2}+3x_{3}+3x_{4}-x_{5}=2
\end{aligned}
\right.
$$
\begin{proof}[解]
	$
	\left[
	\begin{matrix}
	1 & 1 & 1 & 1 & 1 & 1\\
	3 & 2 & 1 & 1 & -3 & 0\\
	0 & 1 & 2 & 2 & 6 & 3\\
	5 & 4 & 3 & 3 & -1 & 2
	\end{matrix}
	\right]
	\xrightarrow{R_{2}-3R_{1};R_{4}-5R_{1}}
	\left[
	\begin{matrix}
	1 & 1 & 1 & 1 & 1 & 1\\
	0 & 1 & 2 & 2 & 6 & 3\\
	0 & 1 & 2 & 2 & 6 & 3\\
	0 & 1 & 2 & 2 & 6 & 3
	\end{matrix}
	\right]\\
	\xrightarrow{}
	\left[
	\begin{matrix}
	1 & 0 & -1 & -1 & -5 & -2\\
	0 & 1 & 2 & 2 & 6 & 3\\
	0 & 0 & 0 & 0 & 0 & 0\\
	0 & 0 & 0 & 0 & 0 & 0
	\end{matrix}
	\right]\\
	$
	得到对应齐次方程的一个基础解系为$(1,-2,1,0,0)^{T},(1,-2,0,1,0)^{T},(5,-6,0,0,1)^{T}$,得到一个非齐次方程特解为$(-2,3,0,0,0)^{T}$,所以,特解为:
	$$
	\left[
	\begin{matrix}
	x_{1}\\
	x_{2}\\
	x_{3}\\
	x_{4}\\
	x_{5}
	\end{matrix}
	\right] = 
	\left[
	\begin{matrix}
	-2\\
	3\\
	0\\
	0\\
	0
	\end{matrix}
	\right] + k_{1}
	\left[
	\begin{matrix}
	1\\
	-2\\
	1\\
	0\\
	0
	\end{matrix}
	\right] + k_{2}
	\left[
	\begin{matrix}
	1\\
	-2\\
	0\\
	1\\
	0
	\end{matrix}
	\right] + k_{3}
	\left[
	\begin{matrix}
	5\\
	-6\\
	0\\
	0\\
	1
	\end{matrix}
	\right]
	$$
\end{proof}
\noindent 30.已知:向量组I可用向量组II线性表出,向量组II可用向量组III线性表出,求出:向量组I可用向量组III线性表出.
\begin{proof}[证]
	不妨令向量组I为($\alpha_{1},\alpha_{2},...,\alpha_{m}$).\\
	向量组II为($\beta_{1},\beta_{2},...,\beta_{n}$).\\
	向量组III为($\gamma_{1},\gamma_{2},...,\gamma_{k}$).\\
	而向量组I可用向量组II线性表出,所以存在$n \times m$矩阵A使得:
	$$
	[\alpha_{1},\alpha_{2},...,\alpha_{m}] = [\beta_{1},\beta_{2},...,\beta_{n}]A
	$$
	向量组II可用向量组III线性表出,所以存在$k \times n$矩阵B使得:
	$$
	[\beta_{1},\beta_{2},...,\beta_{n}] = [\gamma_{1},\gamma_{2},...,\gamma_{k}]B
	$$
	所以:
	$$
	[\alpha_{1},\alpha_{2},...,\alpha_{m}] = [\beta_{1},\beta_{2},...,\beta_{n}]A
	$$
	$$
	= [\gamma_{1},\gamma_{2},...,\gamma_{k}]BA
	$$
	其中$BA$为$k \times m$矩阵,则向量组I可用向量组III线性表出.
\end{proof}
\noindent 31.设向量组$\alpha_{1}, \alpha_{2}, \cdots, \alpha_{n}$线性无关. 证明:当且仅当$n$为奇数时,向量组$\alpha_{1}+\alpha_{2},\alpha_{2}+\alpha_{3},\cdots , \alpha_{n-1}+\alpha_{n}, \alpha_{n}+\alpha_{1}$也线性无关.
\begin{proof}[证]
	假设向量组$\alpha_{1}+\alpha_{2},\alpha_{2}+\alpha_{3},\cdots , \alpha_{n-1}+\alpha_{n}, \alpha_{n}+\alpha_{1}$线性相关,那必然存在不全为0的实数$k_{1}, k_{2}, k_{3}, \cdots, k_{n}$,使得
	$$k_{1}(\alpha_{1}+\alpha_{2})+k_{2}(\alpha_{2}+\alpha_{3})+\cdots + k_{n-1}(\alpha_{n-1}+\alpha_{n})+ k_{n}(\alpha_{n}+\alpha_{1})=0$$
	整理得
	$$
	(k_{1}+k_{n})\alpha_{1} + (k_{1}+k_{2})\alpha_{2} + \cdots + (k_{n-1}+k_{n})\alpha_{n} = 0
	$$
	因为$\alpha_{1}, \alpha_{2}, \cdots, \alpha_{n}$线性无关,所以我们得到方程组
	$$
	\left\{
	\begin{aligned}
	k_{1}+k_{n} & = 0 \\
	k_{1}+k_{2} & = 0 \\
	\vdots & \\
	k_{n-1}+k_{n} & = 0
	\end{aligned}
	\right.
	$$
	系数矩阵为
	$$
	\left[
	\begin{matrix}
	1 & 0 & 0 & \cdots & 0 & 1 \\
	1 & 1 & 0 & \cdots & 0 & 0 \\
	0 & 1 & 1 & \cdots & 0 & 0 \\
	\vdots & \vdots & \vdots & \ddots & \vdots & \vdots \\
	0 & 0 & 0 & \cdots & 1 & 0 \\
	0 & 0 & 0 & \cdots & 1 & 1
	\end{matrix}
	\right]
	$$
	1)$n$为偶数
	$$
	\xrightarrow{R_{2}-R_{1}, R_{3}-R_{2}, \cdots, R_{n}-R_{n-1}}
	\left[
	\begin{matrix}
	1 & 0 & 0 & \cdots & 0 & 1 \\
	0 & 1 & 0 & \cdots & 0 & -1 \\
	0 & 0 & 1 & \cdots & 0 & 1 \\
	\vdots & \vdots & \vdots & \ddots & \vdots & \vdots \\
	0 & 0 & 0 & \cdots & 1 & 1 \\
	0 & 0 & 0 & \cdots & 0 & 0
	\end{matrix}
	\right]
	$$
	方程组有非零解,所以$\alpha_{1}+\alpha_{2},\alpha_{2}+\alpha_{3},\cdots , \alpha_{n-1}+\alpha_{n}, \alpha_{n}+\alpha_{1}$线性相关.\\
	2)$n$为奇数
	$$
	\xrightarrow{R_{2}-R_{1}, R_{3}-R_{2}, \cdots, R_{n}-R_{n-1}}
	\left[
	\begin{matrix}
	1 & 0 & 0 & \cdots & 0 & 1 \\
	0 & 1 & 0 & \cdots & 0 & -1 \\
	0 & 0 & 1 & \cdots & 0 & 1 \\
	\vdots & \vdots & \vdots & \ddots & \vdots & \vdots \\
	0 & 0 & 0 & \cdots & 1 & -1 \\
	0 & 0 & 0 & \cdots & 0 & 2
	\end{matrix}
	\right]
	$$
	方程组只有零解,所以$\alpha_{1}+\alpha_{2},\alpha_{2}+\alpha_{3},\cdots , \alpha_{n-1}+\alpha_{n}, \alpha_{n}+\alpha_{1}$线性无关.
\end{proof}
\noindent 32.设向量$\alpha_{1}, \alpha_{2},\cdots ,\alpha_{n}$线性无关,而向量组$\alpha_{1}, \alpha_{2},\cdots ,\alpha_{n},\beta, \gamma$线性相关.证明:或者$\beta$与$\gamma$中至少有一个可由$\alpha_{1}, \alpha_{2},\cdots ,\alpha_{n}$线性表出,或者向量组$\alpha_{1}, \alpha_{2},\cdots ,\alpha_{n},\beta$与向量组$\alpha_{1}, \alpha_{2},\cdots ,\alpha_{n},\gamma$可相互线性表出.
\begin{proof}[证]
	因为$\alpha_{1}, \alpha_{2},\cdots ,\alpha_{n},\beta, \gamma$线性相关,所以存在不全为0得数$k_{1}, k_{2}, \cdots ,k_{n}, r, s$使得
	$$
	k_{1}\alpha_{1}+k_{2}\alpha_{2}+ \cdots +k_{n}\alpha_{n} + r\beta+ s\gamma = 0
	$$
	1)$r=0, s=0$这是不可能的,因为$\alpha_{1}, \alpha_{2},\cdots ,\alpha_{n}$线性无关.\\
	2)$r \neq 0, s=0$,此时$\beta$可由$\alpha_{1}, \alpha_{2},\cdots ,\alpha_{n}$线性表出
	$$
	\beta = -\frac{1}{r}(k_{1}\alpha_{1}+k_{2}\alpha_{2}+ \cdots +k_{n}\alpha_{n})
	$$
	3)$r = 0, s \neq 0$,此时$\gamma$可由$\alpha_{1}, \alpha_{2},\cdots ,\alpha_{n}$线性表出
	$$
	\gamma = -\frac{1}{s}(k_{1}\alpha_{1}+k_{2}\alpha_{2}+ \cdots +k_{n}\alpha_{n})
	$$
	3)$r \neq 0, s \neq 0$,此时$\beta, \gamma$都可由$\alpha_{1}, \alpha_{2},\cdots ,\alpha_{n}$线性表出
	$$
	\beta = -\frac{1}{r}(k_{1}\alpha_{1}+k_{2}\alpha_{2}+ \cdots +k_{n}\alpha_{n}+s\gamma)
	$$
	$$
	\gamma = -\frac{1}{s}(k_{1}\alpha_{1}+k_{2}\alpha_{2}+ \cdots +k_{n}\alpha_{n}+r\beta)
	$$
	而$\alpha_{1}, \alpha_{2},\cdots ,\alpha_{n}$本身可由$\alpha_{1}, \alpha_{2},\cdots ,\alpha_{n}$线性表出,所以,向量组$\alpha_{1}, \alpha_{2},\cdots ,\alpha_{n},\beta$与向量组$\alpha_{1}, \alpha_{2},\cdots ,\alpha_{n},\gamma$可相互线性表出.
\end{proof}
\noindent 33.设$\beta_{1} = \alpha_{2} + \alpha_{3} + \cdots + \alpha_{n}, \beta_{2} = \alpha_{1} + \alpha_{3} + \cdots + \alpha_{n}, \beta_{n} = \alpha_{1} + \alpha_{2} + \cdots + \alpha_{n-1}$,其中$m > 1$.证明:向量组$\alpha_{1}, \alpha_{2}, \cdots, \alpha_{n}$与向量组$\beta_{1}, \beta_{2}, \cdots, \beta_{n}$可相互线性表出.
\begin{proof}[证]
	注意到
	$$
	[\beta_{1}, \beta_{2}, \cdots, \beta_{n}] = [\alpha_{1}, \alpha_{2}, \cdots, \alpha_{n}]\left[
	\begin{matrix}
	0 & 1 & 1 & \cdots & 1 & 1 \\
	1 & 0 & 1 & \cdots & 1 & 1 \\
	1 & 1 & 0 & \cdots & 1 & 1 \\
	\vdots & \vdots & \vdots & \ddots & \vdots & \vdots \\
	1 & 1 & 1 & \cdots & 0 & 1\\
	1 & 1 & 1 & \cdots & 1 & 0
	\end{matrix}
	\right]
	$$
	令$A = \left[
	\begin{matrix}
	0 & 1 & 1 & \cdots & 1 & 1 \\
	1 & 0 & 1 & \cdots & 1 & 1 \\
	1 & 1 & 0 & \cdots & 1 & 1 \\
	\vdots & \vdots & \vdots & \ddots & \vdots & \vdots \\
	1 & 1 & 1 & \cdots & 0 & 1\\
	1 & 1 & 1 & \cdots & 1 & 0
	\end{matrix}
	\right]$,$A$是满秩矩阵,所以$A$可逆,所以
	$$
	[\beta_{1}, \beta_{2}, \cdots, \beta_{n}]A^{-1} = [\alpha_{1}, \alpha_{2}, \cdots, \alpha_{n}]
	$$
	所以,$\alpha_{1}, \alpha_{2}, \cdots, \alpha_{n}$能由$\beta_{1}, \beta_{2}, \cdots, \beta_{n}$线性表出.
\end{proof}
\noindent 34.设向量组$\alpha_{1},\alpha_{2},...,\alpha_{n}$与$\beta_{1},\beta_{2},...,\beta_{n}$满足:
$$
\alpha_{i} = \left[
\begin{matrix}
x_{1i}\\
x_{2i}\\
\vdots\\
x_{ri}
\end{matrix}
\right], \beta_{i} = \left[
\begin{matrix}
x_{1i}\\
x_{2i}\\
\vdots\\
x_{ri}\\
x_{(r+1)i}\\
\vdots\\
x_{mi}
\end{matrix}
\right],i=1,2,...,n
$$
证明:若$\beta_{1},\beta_{2},...,\beta_{n}$线性相关,则$\alpha_{1},\alpha_{2},...,\alpha_{n}$线性相关;若$\alpha_{1},\alpha_{2},...,\alpha_{n}$线性无关,则$\beta_{1},\beta_{2},...,\beta_{n}$线性无关.
\begin{proof}[证]
	令$[\alpha_{1},\alpha_{2},...,\alpha_{n}] = A$,$A$为$r \times n$的矩阵.\\
	则$[\beta_{1},\beta_{2},...,\beta_{n}]$可用分块矩阵$\left[
	\begin{matrix}
	A\\
	B
	\end{matrix}
	\right]$表示.\\
	而$\beta_{1},\beta_{2},...,\beta_{n}$线性相关.\\
	所以,$\left[
	\begin{matrix}
	A\\
	B
	\end{matrix}
	\right]x=0$有非零解.即$Ax=0$有非零解.\\
	所以,$\alpha_{1},\alpha_{2},...,\alpha_{n}$线性相关.\\
	后者是前者的逆否命题,所以也成立.即若$\alpha_{1},\alpha_{2},...,\alpha_{n}$线性无关,则$\beta_{1},\beta_{2},...,\beta_{n}$线性无关.
\end{proof}
\noindent 35.给定$n$个非零的数$a_{1}, a_{2}, \cdots, a_{n}$,求下面向量组的秩:
$$
\eta_{1} = [1+a_{1}, 1, \cdots, 1],\eta_{2} = [1, 1+a_{2}, \cdots, 1],\cdots, \eta_{n} = [1, 1, \cdots, 1+a_{n}]
$$
\begin{proof}[解]
	$[\eta_{1}, \eta_{2}, \cdots, \eta_{n}] = \left[
	\begin{matrix}
	1+a_{1} & 1 & 1 & \cdots & 1 & 1 \\
	1 & 1+a_{2} & 1 & \cdots & 1 & 1 \\
	1 & 1 & 1+a_{3} & \cdots & 1 & 1 \\
	\vdots & \vdots & \vdots & \ddots & \vdots & \vdots \\
	1 & 1 & 1 & \cdots & 1+a_{n-1} & 1\\
	1 & 1 & 1 & \cdots & 1 & 1+a_{n}
	\end{matrix}
	\right]$
	$$
	\xrightarrow{R_{2}-R_{1}, R_{3}-R_{1}, ...,R_{n}-R_{1}}
	\left[
	\begin{matrix}
	1+a_{1} & 1 & 1 & \cdots & 1 & 1 \\
	-a_{1} & a_{2} & 0 & \cdots & 0 & 0 \\
	-a_{1} & 0 & a_{3} & \cdots & 0 & 0 \\
	\vdots & \vdots & \vdots & \ddots & \vdots & \vdots \\
	-a_{1} & 0 & 0 & \cdots & a_{n-1} & 0\\
	-a_{1} & 0 & 0 & \cdots & 0 & a_{n}
	\end{matrix}
	\right]
	$$
	$$
	\xrightarrow{C_{1}+\frac{a_{1}}{a_{2}}+\frac{a_{1}}{a_{3}}+...+\frac{a_{1}}{a_{n}}}
	\left[
	\begin{matrix}
	1+a_{1}+ \frac{a_{1}}{a_{2}}+\frac{a_{1}}{a_{3}}+...+\frac{a_{1}}{a_{n}}& 1 & 1 & \cdots & 1 & 1 \\
	0 & a_{2} & 0 & \cdots & 0 & 0 \\
	0 & 0 & a_{3} & \cdots & 0 & 0 \\
	\vdots & \vdots & \vdots & \ddots & \vdots & \vdots \\
	0 & 0 & 0 & \cdots & a_{n-1} & 0\\
	0 & 0 & 0 & \cdots & 0 & a_{n}
	\end{matrix}
	\right]
	$$
	所以,当$1+a_{1}+ \frac{a_{1}}{a_{2}}+\frac{a_{1}}{a_{3}}+...+\frac{a_{1}}{a_{n}} = 0$,即$1+\frac{1}{a_{1}}+ \frac{1}{a_{2}}+\frac{1}{a_{3}}+...+\frac{1}{a_{n}}=0$,秩为$n-1$.\\
	当$1+a_{1}+ \frac{a_{1}}{a_{2}}+\frac{a_{1}}{a_{3}}+...+\frac{a_{1}}{a_{n}} \neq 0$,即$1+\frac{1}{a_{1}}+ \frac{1}{a_{2}}+\frac{1}{a_{3}}+...+\frac{1}{a_{n}} \neq 0$,秩为$n$.
\end{proof}
\noindent 36.设$A,B$分别是$m \times n$和$n \times s$的矩阵,且$AB=O$.证明:$rank(A)+rank(B) \leq n$
\begin{proof}[证]
	由$p_{45}$的例2.2.3得$r(A)+r(B) \leq r(AB)+n = n$
\end{proof}
\noindent 37.设$\eta_{1},\eta_{2},...,\eta_{t}$是某一非齐次线性方程组得解.证明:$\mu_{1}\eta_{1}+\mu_{2}\eta_{2}+...+\mu_{t}\eta_{t}$也是该齐次线性方程组的解的充要条件是$\mu_{1}+\mu_{2}+...+\mu_{t}=1$.
\begin{proof}[证]
	令$\eta_{1},\eta_{2},...,\eta_{n}$是$Ax=b$的解.\\
	则令$A(\mu_{1}\eta_{1}+\mu_{2}\eta_{2}+...+\mu_{t}\eta_{t})=b$\\
	$\Leftrightarrow \mu_{1}A\eta_{1}+\mu_{2}A\eta_{2}+...+\mu_{t}A\eta_{t}=b$\\
	$\Leftrightarrow \mu_{1}b+\mu_{2}b+...+\mu_{t}b=b$\\
	$\Leftrightarrow (\mu_{1}+\mu_{2}+...+\mu_{t})b=b$\\
	$\Leftrightarrow \mu_{1}+\mu_{2}+...+\mu_{t} = 1$ \\
	充要性得证.
\end{proof}
\noindent 38.证明:方程组
$$
\left\{
\begin{aligned}
a_{11}x_{1}+a_{12}x_{2}+\cdots+a_{1n}x_{n}=0 \\
a_{21}x_{1}+a_{22}x_{2}+\cdots+a_{2n}x_{n}=0 \\
\cdots & \\
a_{m1}x_{1}+a_{m2}x_{2}+\cdots+a_{mn}x_{n}=0 
\end{aligned}
\right.
$$
的解全是方程$b_{1}x_{1}+b_{2}x_{2}+\cdots+b_{n}x_{n}=0$的解的充要条件是:向量$\beta = [b_{1}, b_{2}, ..., b_{n}]$可由向量组$\alpha_{1}, \alpha_{2},...,\alpha_{m}$线性表出,其中
$$
\alpha_{i}=[a_{i1}, a_{i2},...,a_{in}], i=1,2,...,m
$$
\begin{proof}[证]
	1)必要性:\\
	令系数矩阵为$A =
	\left[
	\begin{matrix}
	\alpha_{1} \\
	\alpha_{2} \\
	\vdots\\
	\alpha_{m}
	\end{matrix}
	\right]
	$
	则$Ax=0$的解,全是$\beta x=0$的解.\\
	所以$Ax=0$与$\left[
	\begin{matrix}
	A \\
	\beta
	\end{matrix}
	\right]x = 0$同解.\\
	这意味着$r(A) = r(\left[
	\begin{matrix}
	A \\
	\beta
	\end{matrix}
	\right])$, $r(A^{T})=r([A^{T}|\beta^{T}])$\\
	即向量$\beta = [b_{1}, b_{2}, ..., b_{n}]$可由向量组$\alpha_{1}, \alpha_{2},...,\alpha_{m}$线性表出.\\
	2)充分性:\\
	向量$\beta = [b_{1}, b_{2}, ..., b_{n}]$可由向量组$\alpha_{1}, \alpha_{2},...,\alpha_{m}$线性表出.\\
	则存在不全为0的数$k_{1}, k_{2},...,k_{m}$
	使得
	$$
	\beta = k_{1}\alpha_{1}+k_{2}\alpha_{2}+\cdots+k_{m}\alpha_{m}
	$$
	假设$x$为$Ax=0$的解\\
	则$\alpha_{1}x=0, \alpha_{2}x = 0, \cdots , \alpha_{m}x = 0$.\\
	所以$k_{1}\alpha_{1}x=0, k_{2}\alpha_{2}x = 0, \cdots , k_{m}\alpha_{m}x = 0$\\
	$\beta x=(k_{1}\alpha_{1}+k_{2}\alpha_{2}+\cdots+k_{m}\alpha_{m})x = k_{1}\alpha_{1}x+k_{2}\alpha_{2}x+\cdots +k_{m}\alpha_{m}x = 0$\\
	我们得到$x$也是$\beta x = 0$的解.
\end{proof}
\noindent 39.求下列矩阵的逆矩阵(只有答案)\\
\noindent (1)$
\left[
\begin{matrix}
1 & 1 & 1 & 1\\
1 & 1 & -1 & -1\\
1 & -1 & 1 & -1\\
1 & -1 & -1 & 1
\end{matrix}
\right]$;(2)$
\left[
\begin{matrix}
5 & 2 & 0 & 0\\
2 & 1 & 0 & 0\\
0 & 0 & 1 & -2\\
0 & 0 & 1 & 1
\end{matrix}
\right]$;(3)$
\left[
\begin{matrix}
1 & a & a^{2} & a^{3}\\
0 & 1 & a & a^{2}\\
0 & 0 & 1 & a\\
0 & 0 & 0 & 1
\end{matrix}
\right]$;\\
(4)
$
\left[
\begin{matrix}
0 & a_{1} & 0 & \cdots & 0\\
0 & 0 & a_{2} & \cdots & 0\\
\vdots & \vdots & \vdots & \ddots & \vdots\\ 
0 & 0 & 0 & \cdots & a_{n-1}\\
a_{n} & 0 & 0 & \cdots & 0
\end{matrix}
\right]
$
\begin{proof}[解]
	(1)
	$$
	A^{-1}=\frac{1}{4}A =  \frac{1}{4}\left[
	\begin{matrix}
	1 & 1 & 1 & 1\\
	1 & 1 & -1 & -1\\
	1 & -1 & 1 & -1\\
	1 & -1 & -1 & 1
	\end{matrix}
	\right]
	$$
	(2)
	$$
	A^{-1} = \left[
	\begin{matrix}
	1 & -2 & 0 & 0\\
	-2 & 5 & 0 & 0\\
	0 & 0 & \frac{1}{3} & \frac{2}{3}\\
	0 & 0 & -\frac{1}{3} & \frac{1}{3}
	\end{matrix}
	\right]
	$$
	(3)
	$$
	A^{-1} = \left[
	\begin{matrix}
	1 & -a & 0 & 0\\
	0 & 1 & -a & 0\\
	0 & 0 & 1 & -a\\
	0 & 0 & 0 & 1
	\end{matrix}
	\right]
	$$
	(4)
	$$
	A^{-1} = \left[
	\begin{matrix}
	0 & 0 & 0 & \cdots & 0 & \frac{1}{a_{n}}\\
	\frac{1}{a_{1}} & 0 & 0 & \cdots & 0 & 0\\
	0 & \frac{1}{a_{2}} & 0 & \cdots & 0 & 0\\
	\vdots & \vdots & \vdots & \ddots & \vdots & \vdots\\ 
	0 & 0 & 0 & \cdots & \frac{1}{a_{n-1}} & 0
	\end{matrix}
	\right]
	$$
\end{proof}
\noindent 40.解下列矩阵方程:(只有答案)\\
(1)$\left[
\begin{matrix}
1 & 1 & 1\\
0 & 1 & 1\\
0 & 0 & 1
\end{matrix}
\right]X= \left[
\begin{matrix}
5 & 6\\
3 & 4\\
1 & 2
\end{matrix}
\right]$\\
(2)$X\left[
\begin{matrix}
1 & 2 & -3\\
3 & 2 & -4\\
2 & -1 & 0
\end{matrix}
\right]= \left[
\begin{matrix}
1 & -3 & 0\\
10 & 2 & 7\\
10 & 7 & 8
\end{matrix}
\right]$
\begin{proof}[解]
	(1)
	$$X = \left[
	\begin{matrix}
	2 & 2\\
	2 & 2\\
	1 & 2
	\end{matrix}
	\right]$$
	(2)
	$$
	X = \left[
	\begin{matrix}
	20 & -15 & 13\\
	-105 & 77 & -58\\
	-152 & 112 & -87
	\end{matrix}
	\right]
	$$
\end{proof}
\noindent 41.设$A,B,C$均为$n$阶方阵,若$ABC=I$,则下列乘积:$ACB,BAC,BCA,CAB,CBA$中哪些必等于单位阵$I$.
\begin{proof}[证]
	$ABC = I$,则$A, B, C$均可逆.\\
	$BCA = A^{-1}ABCA = A^{-1}IA = I$\\
	$CAB = B^{-1}BCAB = B^{-1}IB = I$\\
	其他均不一定等于$I$.
\end{proof}
\noindent 46.设$A,B$均为$n$阶可逆矩阵.证明:如果$A+B$可逆,则$A^{-1}+B^{-1}$也可逆,并求其逆矩阵.
\begin{proof}[证]
	注意到$A^{-1}(A+B)B^{-1} = A^{-1}AB^{-1}+A^{-1}BB^{-1} = B^{-1}+A^{-1}.$\\
	而$A, B, A+B$均可逆,所以$A^{-1}+B^{-1}$也可逆.\\
	$(A^{-1}+B^{-1})^{-1} = (A^{-1}(A+B)B^{-1})^{-1} = B(A+B)^{-1}A.$
\end{proof}
\noindent 47.(1)设$A$为$n$阶方阵,满足$A^{3}+2A^{2}-2A-I_{n}=O$,证明:$A+I_{n}$是可逆矩阵,并求其逆矩阵;\\
\noindent (2)设$A,B$为$n$阶方阵,且$A-I_{n}$和$B$可逆.证明:若$(A-I_{n})^{-1} = (B-I_{n})^{T}$,则有$A$可逆;\\
\noindent (3)设$A$为$n$阶方阵,满足$A^{2}+A-6I_{n}=O$,证明:$A, A+I_{n}, A+4I_{n}$是可逆矩阵,并求其逆矩阵.\\
\noindent (1)
\begin{proof}[证]
	$A^{3}+2A^{2}-2A-I_{n}=O$\\
	$\Rightarrow A^{3}+2A^{2}-2A-3I_{n}=-2I_{n}$\\
	$\Rightarrow (A+I_{n})(A^{2}+A-3I_{n})=-2I_{n}$\\
	$\Rightarrow A+I_{n}$可逆.\\
	$\Rightarrow(A+I_{n})^{-1} = -\frac{1}{2}(A^{2}+A-3I_{n})$
\end{proof}
\noindent (2)
\begin{proof}[证]
	$(A-I_{n})^{-1} = (B-I_{n})^{T}$\\
	$\Rightarrow (A-I_{n})^{-1}(A-I_{n}) = (B-I_{n})^{T}(A-I_{n})$\\
	$\Rightarrow I_{n} = (B^{T}-I_{n})(A-I_{n})$\\
	$\Rightarrow I_{n} = B^{T}A-A-B^{T}+I_{n}$\\
	$\Rightarrow B^{T}A-A-B^{T} = O$\\
	$\Rightarrow B^{T}(A-I_{n}) = A$\\
	而$B, A-I_{n}$可逆,所以$A$可逆.
\end{proof}
\noindent (3)
\begin{proof}[证]
	$A^{2}+A-6I_{n}=O$\\
	$\Rightarrow A^{2}+A=6I_{n}$\\
	$\Rightarrow A(A+I_{n})=6I_{n}$\\
	所以,$A,A+I_{n}$都可逆.\\
	$A^{-1} = \frac{1}{6}(A+I_{n})$\\
	$(A+I_{n})^{-1} = \frac{1}{6}A$\\
	$A^{2}+A-12I_{n}=-6I_{n}$\\
	$(A-3I_{n})(A+4I_{n})=-6I_{n}$\\
	所以,$A+4I_{n}$可逆.\\
	$(A+4I_{n})^{-1} = -\frac{1}{6}(A-3I_{n})$
\end{proof}
\noindent 48.已知$A,B$为3阶方阵,且满足$2A^{-1}B=B-4I$.\\
\noindent (1)证明:矩阵$A-2I$可逆;\\
\noindent (2)若$B = \left[
\begin{matrix}
1 & -2 & 0\\
1 & 2 & 0\\
0 & 0 & 2
\end{matrix}
\right]$求$A$.
\noindent (1)
\begin{proof}[证]
	$2A^{-1}B=B-4I$\\
	$\Rightarrow 2B = AB-4A$\\
	$\Rightarrow AB-2B=4A$\\
	$\Rightarrow (A-2I)B=4A$\\
	$\Rightarrow (A-2I)BA^{-1} = 4I$\\
	所以,$A-2I$可逆.
\end{proof}
\noindent (2)
\begin{proof}[解]
	$A=\left[
	\begin{matrix}
	0 & 2 & 0\\
	-1 & -1 & 0\\
	0 & 0 & -2
	\end{matrix}
	\right]$
\end{proof}
\noindent 49.设$\alpha$是$n$维非零列向量,记$A=I_{n}-\alpha\alpha^{T}$.证明:\\
\noindent (1)$A^{2}=A$的充分必要条件为$\alpha^{T}\alpha=1$;\\
\noindent (2)当$\alpha^{T}\alpha=1$时,$A$是不可逆矩阵;\\
\begin{proof}[证]
	(1)\\
	充分性:\\
	$A^{2} = (I_{n}-\alpha\alpha^{T})(I_{n}-\alpha\alpha^{T}) = I_{n}-2\alpha\alpha^{T}+\alpha\alpha^{T}\alpha\alpha^{T} = I_{n}-\alpha\alpha^{T} = A$\\
	必要性:\\
	$A^{2} = A$\\
	$\Rightarrow (I_{n}-\alpha\alpha^{T})(I_{n}-\alpha\alpha^{T}) = I_{n}-\alpha\alpha^{T}$\\
	$\Rightarrow I_{n}-(2-\alpha^{T}\alpha)\alpha\alpha^{T} =  I_{n}-\alpha\alpha^{T} $\\
	$\Rightarrow 2-\alpha^{T}\alpha = 1$\\
	$\Rightarrow \alpha^{T}\alpha = 1$\\
	(2)\\
	由(1)知$A^{2}=A$,即$A(I_{n}-A)=0$,即$A(\alpha\alpha^{T}) = 0$.\\
	所以,$r(A)+r(\alpha\alpha^{T}) \leq n$\\
	而$\alpha$是非零向量,所以$\alpha\alpha^{T}$为非零矩阵.\\
	所以$r(\alpha\alpha^{T}) \geq 1$(事实上$r(\alpha\alpha^{T}) = 1$)\\
	所以$r(A) \leq n-1$,即$A$为不可逆矩阵.
\end{proof}
\noindent 50.
\begin{proof}[证]
	$A^{T} = (I_{n}-P^{T}(PP^{T})^{-1}P)^{T}$\\
	$=I_{n}-P^{T}((PP^{T})^{T})^{-1}P$\\
	$=I_{n}-P^{T}(PP^{T})^{-1}P = A$\\
	所以,$A$是对称矩阵.\\
	$A^{2} = (I_{n}-P^{T}(PP^{T})^{-1}P)(I_{n}-P^{T}(PP^{T})^{-1}P)$\\
	$=I_{n}-2P^{T}(PP^{T})^{-1}P+P^{T}(PP^{T})^{-1}PP^{T}(PP^{T})^{-1}P$\\
	$=I_{n}-2P^{T}(PP^{T})^{-1}P+P^{T}(PP^{T})^{-1}P$\\
	$=I_{n}-P^{T}(PP^{T})^{-1}P = A$\\
	所以,$A^{2}=A$.
\end{proof}
\noindent 51.
\begin{proof}[证]
	证明$A^{T}A=I_{n}$即可.
\end{proof}
\noindent 52.
\begin{proof}[解]
	$A$为对角元素为1或-1的对角矩阵.
\end{proof}
\end{document}

