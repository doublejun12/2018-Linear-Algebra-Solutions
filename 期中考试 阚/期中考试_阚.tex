\documentclass[10pt,a4paper]{report}
\usepackage[utf8]{inputenc}
\usepackage{amsmath}
\usepackage{amsfonts}
\usepackage{amssymb}
\usepackage{graphicx}
\usepackage{hyperref}
\usepackage{amsthm}
\usepackage{enumitem}
\usepackage{xeCJK}
\usepackage{extarrows}

\def\*#1{\mathbf{#1}}
\def\rand{\xleftarrow{\$}}

\title{}
\author{}
\date{}
\begin{document}

\chapter*{期中考试答案}
\noindent 一.\\
\noindent 1.\\
\begin{proof}[解]
	将行列式按第一列展开:\\
	$B_{n} = 
	\left|
	\begin{matrix}
	a+b & ab & 0 & \cdots & 0 & 0 & 0 & 0\\
	1 & a+b & ab & \cdots & 0 & 0 & 0 & 0\\
	\vdots & \vdots & \vdots & \ddots & \vdots & \vdots & \vdots & \vdots \\
	0 & 0 & 0 & \cdots & 1 & a+b & ab & 0 \\
	0 & 0 & 0 & \cdots & 0 & 1 & a+b & ab\\
	0 & 0 & 0 & \cdots & 0 & 0 & 1 & a+b
	\end{matrix}
	\right|\\
	(a+b)
	\left|
	\begin{matrix}
	a+b & ab & \cdots & 0 & 0 & 0 & 0\\
	\vdots & \vdots & \ddots & \vdots & \vdots & \vdots & \vdots \\
	0 & 0 & \cdots & 1 & a+b & ab & 0 \\
	0 & 0 & \cdots & 0 & 1 & a+b & ab\\
	0 & 0 & \cdots & 0 & 0 & 1 & a+b
	\end{matrix}
	\right|\\
	+(-1)^{1+2}
	\left|
	\begin{matrix}
	ab & 0 & \cdots & 0 & 0 & 0 & 0\\
	\vdots & \vdots & \ddots & \vdots & \vdots & \vdots & \vdots \\
	0 & 0 & \cdots & 1 & a+b & ab & 0 \\
	0 & 0 & \cdots & 0 & 1 & a+b & ab\\
	0 & 0 & \cdots & 0 & 0 & 1 & a+b
	\end{matrix}
	\right|\\
	=(a+b)B_{n-1}-abB_{n-2}$\\
	我们可以得到:\\
	$B_{n}-bB_{n-1} = aB_{n-1}-abB_{n-2} = a(B_{n-1}-bB_{n-2})$\\
	$B_{1} = a+b$,
	$B_{2} = \left|
	\begin{matrix}
	a+b & ab\\
	1 & a+b 
	\end{matrix}
	\right| = a^{2}+ab+b^{2}$\\
	$B_{2}-bB_{1} = a^{2}+ab+b^{2}-ab-b^{2}=a^{2}$\\
	所以$B_{n}-bB_{n-1} = a^{n-2}a^{2} = a^{n}$\\
	同理可得$B_{n}-aB_{n-1} = b^{n-2}b^{2} = b^{n}$\\
	当$a \neq b$时:\\
	得到$aB_{n}-bB_{n} = a^{n+1}-b^{n+1}$\\
	即$B_{n} = \frac{a^{n+1}-b^{n+1}}{a-b}$\\
	当$a = b$时:\\
	得到$B_{n}-aB_{n-1} = a^{n}$
	即$\frac{B_{n}}{a^{n}} - \frac{B_{n-1}}{a^{n-1}} = 1$\\
	又$\frac{B_{1}}{a} = 2$, 所以$\frac{B_{n}}{a^{n}} = n+1$, 即$B_{n} = (n+1)a^{n}$.
\end{proof}
\noindent 2.\\
\begin{proof}[解]
	1.没有$x_{i} = 0$\\
	$A_{n} = 
	\left|
	\begin{matrix}
	x_{1} & a & \cdots & a\\
	a & x_{2} & \cdots & a\\
	\vdots & \vdots & \ddots & \vdots\\
	a & a & \cdots & x_{n}
	\end{matrix}
	\right| \xlongequal {R_{2}-R_{1},...,R_{n}-R_{1}}
	\left|
	\begin{matrix}
	x_{1} & a & \cdots & a\\
	a-x_{1} & x_{2}-a & \cdots & 0\\
	\vdots & \vdots & \ddots & \vdots\\
	a-x_{1} & 0 & \cdots & x_{n}-a
	\end{matrix}
	\right|\\
	\xlongequal {C_{1}+\frac{x_{1}-a}{x_{2}-a}C_{2},...,C_{1}+\frac{x_{1}-a}{x_{n}-a}C_{n}}
	\left|
	\begin{matrix}
	x_{1}+\frac{x_{1}-a}{x_{2}-a}a+...+\frac{x_{1}-a}{x_{n}-a}a & a & \cdots & a\\
	a-x_{1} & x_{2}-a & \cdots & 0\\
	\vdots & \vdots & \ddots & \vdots\\
	a-x_{1} & 0 & \cdots & x_{n}-a
	\end{matrix}
	\right|\\
	= (x_{1}+\frac{x_{1}-a}{x_{2}-a}a+...+\frac{x_{1}-a}{x_{n}-a}a)(x_{2}-a)(x_{3}-a)...(x_{n}-a)
	$\\
	2.存在一个$x_{i} = 0$, 其他不为$0$.\\
	$A_{n} = \left|
	\begin{matrix}
	x_{1} & a & \cdots & a\\
	a & x_{2} & \cdots & a\\
	\vdots & \vdots & \ddots & \vdots\\
	a & a & \cdots & a \\
	\vdots & \vdots & \ddots & \vdots\\
	a & a & \cdots & x_{n}
	\end{matrix}
	\right| \xlongequal {R_{1}-R_{i},...,R_{n}-R_{i}}
	\left|
	\begin{matrix}
	x_{1}-a & 0 & \cdots & 0\\
	0 & x_{2}-a & \cdots & 0\\
	\vdots & \vdots & \ddots & \vdots\\
	a & a & \cdots & a \\
	\vdots & \vdots & \ddots & \vdots\\
	0 & 0 & \cdots & x_{n}-a
	\end{matrix}
	\right|\\
	\xlongequal {R_{i}-\frac{a}{x_{1}-a}R_{1},...,R_{i}-\frac{a}{x_{n}-a}R_{n}}
	\left|
	\begin{matrix}
	x_{1}-a & 0 & \cdots & 0 & \cdots & 0\\
	0 & x_{2}-a & \cdots & 0 & \cdots & 0\\
	\vdots & \vdots & \ddots & \vdots & \ddots & \vdots\\
	0 & 0 & \cdots & a & \cdots & 0 \\
	\vdots & \vdots & \ddots & \vdots& \ddots & \vdots\\
	0 & 0 & \cdots & 0 & \cdots & x_{n}-a
	\end{matrix}
	\right|\\
	=(x_{1}-a)(x_{2}-a)...(x_{i-1}-a)a(x_{i+1}-a)...(x_{n}-a)
	$\\
	2.存在两个以上$x_{i} = 0$.\\
	则$A_{n} = 0$\\
\end{proof}
\noindent 二.\\
\begin{proof}[解]
	(1)\\
	$AA^{*}=|A|E\\
	\Rightarrow |AA^{*}|=||A|E|=|A|^{n}\\
	\Rightarrow |A||A^{*}| = |A|^{n}\\
	\Rightarrow |A^{*}| = |A|^{n-1} = 3^{n-1}$\\
	(2)\\
	$AA^{*}=|A|E$\\
	若A可逆,则$A^{*}=|A|A^{-1}$\\
	所以,$|(\frac{1}{4}A)^{-1}-2A^{*}|\\
	=|4A^{-1} - 2A^{*}|\\
	=|4A^{-1} - 2|A|A^{-1}|\\
	=|(4-2|A|)A^{-1}|\\
	=|(-2)A^{-1}|\\
	=(-2)^{n}|A|^{-1}\\
	=\frac{(-2)^{n}}{3}
	$
\end{proof}
\noindent 三.\\
\begin{proof}[解]
	$[\alpha_{1}, \alpha_{2}, \alpha_{3}, \alpha_{4}] = \left[
	\begin{matrix}
	1 & 4 & 2 & 1\\
	2 & 3 & -1 & -3\\
	-1 & 0 & 2 & 3\\
	1 & -2 & -4 & -5\\
	1 & 5 & 3 & 2
	\end{matrix}
	\right]\\
	\xrightarrow{R_{2}-2R_{1},R_{3}+R_{1},R_{4}-R_{1},R_{5}-R_{1}}
	\left[
	\begin{matrix}
	1 & 4 & 2 & 1\\
	0 & -5 & -5 & -5\\
	0 & 4 & 4 & 4\\
	0 & -6 & -6 & -6\\
	0 & 1 & 1 & 1
	\end{matrix}
	\right] \xrightarrow{} \left[
	\begin{matrix}
	1 & 4 & 2 & 1\\
	0 & 1 & 1 & 1\\
	0 & 0 & 0 & 0\\
	0 & 0 & 0 & 0\\
	0 & 0 & 0 & 0
	\end{matrix}
	\right]$\\
	所以,向量组的秩为2,极大线性无关组可以为$\alpha_{1},\alpha_{2}$.\\
\end{proof}
\noindent 四.\\
\begin{proof}[解]
	$[A|b] = \left[
	\begin{matrix}
	2 & 1 & -1 & -1 & 1 & 4\\
	1 & -1 & 1 & 1 & -2 & 1\\
	3 & 3 & -3 & -3 & 4 & 7\\
	4 & 5 & -5 & -5 & 7 & 10
	\end{matrix}
	\right] \xrightarrow{R_{1}-2R_{2}, R_{3}-3R_{2}, R_{4}-4R_{2}}
	\left[
	\begin{matrix}
	0 & 3 & -3 & -3 & 5 & 2\\
	1 & -1 & 1 & 1 & -2 & 1\\
	0 & 6 & -6 & -6 & 10 & 4\\
	0 & 9 & -9 & -9 & 15 & 6
	\end{matrix}
	\right]\\
	\xrightarrow{} \left[
	\begin{matrix}
	1 & -1 & 1 & 1 & -2 & 1\\
	0 & 3 & -3 & -3 & 5 & 2\\
	0 & 0 & 0 & 0 & 0 & 0\\
	0 & 0 & 0 & 0 & 0 & 0
	\end{matrix}
	\right] \xrightarrow{3R_{1}+R_{2}}
	\left[
	\begin{matrix}
	3 & 0 & 0 & 0 & -1 & 5\\
	0 & 3 & -3 & -3 & 5 & 2\\
	0 & 0 & 0 & 0 & 0 & 0\\
	0 & 0 & 0 & 0 & 0 & 0
	\end{matrix}
	\right]$\\
	我们可以得到对应的齐次方程的基础解系为\\
	$\left[
	\begin{matrix}
	0\\
	1\\
	1\\
	0\\
	0
	\end{matrix}
	\right], \left[
	\begin{matrix}
	0\\
	1\\
	0\\
	1\\
	0
	\end{matrix}
	\right], \left[
	\begin{matrix}
	1/3\\
	-5/3\\
	0\\
	0\\
	1
	\end{matrix}
	\right]$\\
	非齐次方程特解为\\
	$
	\left[
	\begin{matrix}
	5/3\\
	2/3\\
	0\\
	0\\
	0
	\end{matrix}
	\right]
	$\\
	所以通解为\\
	$\left[
	\begin{matrix}
	5/3\\
	2/3\\
	0\\
	0\\
	0
	\end{matrix}
	\right]+k_{1}\left[
	\begin{matrix}
	0\\
	1\\
	1\\
	0\\
	0
	\end{matrix}
	\right]+k_{2} \left[
	\begin{matrix}
	0\\
	1\\
	0\\
	1\\
	0
	\end{matrix}
	\right]+k_{3} \left[
	\begin{matrix}
	1/3\\
	-5/3\\
	0\\
	0\\
	1
	\end{matrix}
	\right](k_{1},k_{2},k_{3}$可取任意实数)
\end{proof}
\noindent 五.\\
\begin{proof}[证]
	(1)$\alpha_{1},...,\alpha_{s}$可由$\beta_{1},...,\beta_{t}$线性表示.\\
	则令\\
	$\alpha_{1} = k_{11}\beta_{1}+k_{21}\beta_{2}+...+k_{t1}\beta_{t}$\\
	\vdots\\
	$\alpha_{s} = k_{1s}\beta_{1}+k_{2s}\beta_{2}+...+k_{ts}\beta_{t}$\\
	用矩阵可表示为$[\alpha_{1},...,\alpha_{s}] = [\beta_{1},...,\beta_{t}]\left[
	\begin{matrix}
	k_{11} & \cdots & k_{1s}\\
	k_{21} & \cdots & k_{2s}\\
	\vdots & \ddots & \vdots\\
	k_{t1} & \cdots & k_{ts}
	\end{matrix}
	\right] = [\beta_{1},...,\beta_{t}]A$\\
	要证$\alpha_{1},...,\alpha_{s}$线性相关,只要证$[\alpha_{1},...,\alpha_{s}]X=0$有非零解,只要证$[\beta_{1},...,\beta_{t}]AX = 0$有非零解.\\
	注意到$r(A) \leq min\{t, s\} = t < s$,所以$AX=0$有非零解,则$[\beta_{1},...,\beta_{t}]AX = 0$也有非零解,结论得证.\\
	(2) $\alpha_{1},...,\alpha_{s}$线性相关\\
	则存在不全为0的数$k_{1},...,k_{s}$,使得\\
	$$k_{1}\alpha_{1}+...+k_{s}\alpha_{s}=0$$\\
	乘上$A$得到\\
	$$A(k_{1}\alpha_{1}+...+k_{s}\alpha_{s}=0)=0$$
	$$\Rightarrow{} k_{1}A\alpha_{1}+...+k_{s}A\alpha_{s}=0$$
	所以,$A\alpha_{1},...,A\alpha_{s}$线性相关.\\
\end{proof}
\noindent 六.
\begin{proof}[证]
	(1)首先我们有$r(A)+r(B) = r(\left[
	\begin{matrix}
	A & O\\
	O & B
	\end{matrix}
	\right])$\\
	$\left[
	\begin{matrix}
	A & O\\
	O & B
	\end{matrix}
	\right] \xrightarrow{R_{1}+R_{2}} \left[
	\begin{matrix}
	A & B\\
	O & B
	\end{matrix}
	\right] \xrightarrow{C_{1}+C_{2}}
	\left[
	\begin{matrix}
	A+B & B\\
	O & B
	\end{matrix}
	\right]$\\
	所以,$r(A)+r(B) = r(\left[
	\begin{matrix}
	A & O\\
	O & B
	\end{matrix}
	\right]) = r(\left[
	\begin{matrix}
	A+B & B\\
	O & B
	\end{matrix}
	\right]) \geq r(A+B)$
	相减同理可得。\\
	(2)$r(\left[
	\begin{matrix}
	A & B\\
	O & B
	\end{matrix}
	\right]) = r(\left[
	\begin{matrix}
	A & O\\
	O & B
	\end{matrix}
	\right]) = r(A)+r(B)$\\
	(3)$\left[
	\begin{matrix}
	A & C\\
	O & B
	\end{matrix}
	\right] \xrightarrow{C_{2}-A^{-1}CC_{1}}
	\left[
	\begin{matrix}
	A & O\\
	O & B
	\end{matrix}
	\right]$\\
	所以,$r(\left[
	\begin{matrix}
	A & C\\
	O & B
	\end{matrix}
	\right]) = r(\left[
	\begin{matrix}
	A & O\\
	O & B
	\end{matrix}
	\right]) = r(A)+r(B)$
\end{proof}
\noindent 七.
\begin{proof}[证]
	(1)$r(A)=r$,则存在初等矩阵$P, Q$, 使得$A = P\left[
	\begin{matrix}
	E_{r} & O\\
	O & O
	\end{matrix}
	\right]Q$\\
	而$E_{r} = E_{11}+E_{22}+...+E_{rr}$($E_{ii}$为$i$行$i$列元素为$1$,其他元素都是$0$的$r$阶矩阵)\\
	所以,$A=P\left[
	\begin{matrix}
	E_{r} & O\\
	O & O
	\end{matrix}
	\right]Q = P(\left[
	\begin{matrix}
	E_{11} & O\\
	O & O
	\end{matrix}
	\right] + ...+ \left[
	\begin{matrix}
	E_{rr} & O\\
	O & O
	\end{matrix}
	\right])Q\\
	=P\left[
	\begin{matrix}
	E_{11} & O\\
	O & O
	\end{matrix}
	\right]Q + ...+ P\left[
	\begin{matrix}
	E_{rr} & O\\
	O & O
	\end{matrix}
	\right]Q$\\
	注意到每个$P\left[
	\begin{matrix}
	E_{ii} & O\\
	O & O
	\end{matrix}
	\right]Q$的秩都是$1$,所以$A$能写成$r$个秩为1的矩阵之和.
	(2)$r(A)=n$,则存在初等矩阵$P, Q$, 使得$A = P\left[
	\begin{matrix}
	E_{n}\\
	O
	\end{matrix}
	\right]Q$\\
	注意到初等矩阵都可逆,所以我们可以取秩为$n$的矩阵$B = Q^{-1}[E_{n}, O]P^{-1}$\\
	$BA = Q^{-1}[E_{n}, O]P^{-1}P\left[
	\begin{matrix}
	E_{n}\\
	O
	\end{matrix}
	\right]Q = E_{n}$\\
	所以这样的$B$存在.
\end{proof}
\noindent 八.
\begin{proof}[证]
	令$k_{1}(e_{1}+e_{2})+k_{2}(e_{2}+e_{3})+...+k_{n}(e_{n}+e_{1})=0$\\
	如果$k_{1},...,k_{n}$只能为0,则线性无关,反之,线性相关.\\
	$k_{1}(e_{1}+e_{2})+k_{2}(e_{2}+e_{3})+...+k_{n}(e_{n}+e_{1})\\
	=(k_{1}+k_{n})e_{1}+(k_{1}+k_{2})e_{2}+...+(k_{n-1}+k_{n})e_{n} = 0 $\\
	而又$e_{1},...,e_{n}$线性无关,所以我们可得到方程组:
	$$
	\left\{
	\begin{aligned}
	k_{1}+k_{n} = 0 \\
	k_{1}+k_{2} = 0 \\
	\vdots \\
	k_{n-1}+k_{n} = 0
	\end{aligned}
	\right.
	$$
	系数矩阵为\\
	$
	A=\left[
	\begin{matrix}
	1 & 0 & 0 & \cdots & 0 & 1\\
	1 & 1 & 0 & \cdots & 0 & 0\\
	0 & 1 & 1 & \cdots & 0 & 0\\
	\vdots & \vdots & \vdots & \ddots & \vdots &\vdots\\
	0 & 0 & 0 & \cdots & 1 & 0\\
	0 & 0 & 0 & \cdots & 1 & 1
	\end{matrix}
	\right]
	$\\
	$A$的行列式为\\
	$|A| = \left|
	\begin{matrix}
	1 & 0 & 0 & \cdots & 0 & 1\\
	1 & 1 & 0 & \cdots & 0 & 0\\
	0 & 1 & 1 & \cdots & 0 & 0\\
	\vdots & \vdots & \vdots & \ddots & \vdots &\vdots\\
	0 & 0 & 0 & \cdots & 1 & 0\\
	0 & 0 & 0 & \cdots & 1 & 1
	\end{matrix}
	\right| = (-1)^{1+1}\left|
	\begin{matrix}
	1 & 0 & \cdots & 0 & 0\\
	1 & 1 & \cdots & 0 & 0\\
	\vdots & \vdots & \ddots & \vdots &\vdots\\
	0 & 0 & \cdots & 1 & 0\\
	0 & 0 & \cdots & 1 & 1
	\end{matrix}
	\right| + (-1)^{1+n}\left|
	\begin{matrix}
	1 & 1 & 0 & \cdots & 0\\
	0 & 1 & 1 & \cdots & 0\\
	\vdots & \vdots & \vdots & \ddots & \vdots\\
	0 & 0 & 0 & \cdots & 1\\
	0 & 0 & 0 & \cdots & 1
	\end{matrix}
	\right|\\
	=1+(-1)^{1+n}$\\
	1.若$n$为偶数,则$1+(-1)^{1+n} = 0$,所以$AX=0$有非零解,所以$e_{1}+e_{2}, e_{2}+e_{3},...,e_{n}+e_{1}$线性相关.\\
	2.若$n$为奇数,则$1+(-1)^{1+n} = 2 \neq 0$,所以$AX=0$有只有零解,所以$e_{1}+e_{2}, e_{2}+e_{3},...,e_{n}+e_{1}$线性无关.\\
\end{proof}
\end{document}