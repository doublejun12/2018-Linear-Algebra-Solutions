\documentclass[10pt,a4paper]{report}
\usepackage[utf8]{inputenc}
\usepackage{amsmath}
\usepackage{amsfonts}
\usepackage{amssymb}
\usepackage{graphicx}
\usepackage{hyperref}
\usepackage{amsthm}
\usepackage{enumitem}
\usepackage{xeCJK}
\usepackage{extarrows}

\def\*#1{\mathbf{#1}}
\def\rand{\xleftarrow{\$}}

\title{}
\author{}
\date{}
\begin{document}
\chapter*{第四章:线性空间与线性变换}
\noindent 2.
\begin{proof}[解]
	令$a_{1}f_{1}+a_{2}f_{2},a_{3}f_{3} = 0$,则$a_{1}+a_{2}(x-1)+a_{3}(x-1)^{2}=0$\\
	$$a_{1}-a_{2}+a_{3}+(a_{2}-2a_{3})x+a_{3}x^{2} = 0$$
	因为$1,x,x^{2}$线性无关,所以
	$$
	\left\{
	\begin{aligned}
	a_{1}-a_{2}+a_{3}=0 \\
	a_{2}-2a_{3}=0 \\
	a_{3}=0 
	\end{aligned}
	\right. 
	$$
	解得$a_{1}=a_{2}=a_{3}=0$,所以$f_{1},f_{2},f_{3}$线性无关.而$P_{2}[x]$的维数为3,所以$f_{1},f_{2},f_{3}$是一组基。
	下面令
	$$a_{1}-a_{2}+a_{3}+(a_{2}-2a_{3})x+a_{3}x^{2} = 5x^{2}+x+3$$
	得
	$$
	\left\{
	\begin{aligned}
	a_{1}-a_{2}+a_{3}=3 \\
	a_{2}-2a_{3}=1 \\
	a_{3}=5 
	\end{aligned}
	\right. 
	$$
	解得$$
	\left\{
	\begin{aligned}
	a_{1}=9 \\
	a_{2}=11 \\
	a_{3}=5 
	\end{aligned}
	\right. 
	$$
	所以$g(x)$ 在此基下的坐标:$[9,11,5]^{T}$.
\end{proof}
\noindent 3.
\begin{proof}[解]
	(1)
	$$\left[
	\begin{matrix}
	1 & 1+x & (1+x)^{2} & (1+x)^{3}
	\end{matrix}
	\right] = \left[
	\begin{matrix}
	1 & x & x^{2} & x^{3}
	\end{matrix}
	\right]
	\left[
	\begin{matrix}
	1 & 1 & 1 & 1\\
	0 & 1 & 2 & 3\\
	0 & 0 & 1 & 3\\
	0 & 0 & 0 & 1
	\end{matrix}
	\right]$$
	所以过渡矩阵为$M=\left[
	\begin{matrix}
	1 & 1 & 1 & 1\\
	0 & 1 & 2 & 3\\
	0 & 0 & 1 & 3\\
	0 & 0 & 0 & 1
	\end{matrix}
	\right]$\\
	(2)
	$f(x)$在基$1,x,x^{2}.x^{3}$下坐标为$[a_{0},a_{1},a_{2},a_{3}]^{T}$,在基$1,1+x,(1+x)^{2},(1+x)^{3}]$下坐标为
	$$M^{-1}\left[
	\begin{matrix}
	a_{0}\\
	a_{1}\\
	a_{2}\\
	a_{3}
	\end{matrix}
	\right] = \left[
	\begin{matrix}
	a_{0}-a_{1}+a_{2}-a_{3}\\
	a_{1}-2a_{2}+3a_{3}\\
	a_{2}-3a_{3}\\
	a_{3}
	\end{matrix}
	\right]$$
\end{proof}
\noindent 6.
\begin{proof}[解]
	(1)
	$$\left[
	\begin{matrix}
	\xi_{1} & \xi_{2} & \xi_{3}
	\end{matrix}
	\right] = \left[
	\begin{matrix}
	\epsilon_{1} & \epsilon_{2} & \epsilon_{3}
	\end{matrix}
	\right]\left[
	\begin{matrix}
	1 & 0 & 1\\
	0 & 1 & 2\\
	1 & 0 & 2
	\end{matrix}
	\right] = \left[
	\begin{matrix}
	\epsilon_{1} & \epsilon_{2} & \epsilon_{3}
	\end{matrix}
	\right]M_{1}$$
	
	$$\left[
	\begin{matrix}
	\eta_{1} & \eta_{2} & \eta_{3}
	\end{matrix}
	\right] = \left[
	\begin{matrix}
	\epsilon_{1} & \epsilon_{2} & \epsilon_{3}
	\end{matrix}
	\right]\left[
	\begin{matrix}
	1 & 1 & 1\\
	0 & 1 & 1\\
	0 & 0 & 1
	\end{matrix}
	\right] = \left[
	\begin{matrix}
	\epsilon_{1} & \epsilon_{2} & \epsilon_{3}
	\end{matrix}
	\right]M_{2}$$
	$M_{1},M_{2}$都可逆,且该线性空间维数为3,所以$\xi_{1} , \xi_{2} , \xi_{3}$以及$\eta_{1} , \eta_{2} , \eta_{3}$都是一组基.\\
	(2)令
	$$\left[
	\begin{matrix}
	\eta_{1} & \eta_{2} & \eta_{3}
	\end{matrix}
	\right] = \left[
	\begin{matrix}
	\xi_{1} & \xi_{2} & \xi_{3}
	\end{matrix}
	\right]A$$
	$$
	\left[
	\begin{matrix}
	\epsilon_{1} & \epsilon_{2} & \epsilon_{3}
	\end{matrix}
	\right]M_{2} = \left[
	\begin{matrix}
	\epsilon_{1} & \epsilon_{2} & \epsilon_{3}
	\end{matrix}
	\right]M_{1}A
	$$
	$$A = M_{1}^{-1}M_{2} = \left[
	\begin{matrix}
	2 & 2 & 1\\
	2 & 3 & 1\\
	-1 & -1 & 0
	\end{matrix}
	\right]$$
\end{proof}
\noindent 7.
\begin{proof}[解]
	(1)
	$$\left[
	\begin{matrix}
	\beta_{1} & \beta_{2} & \beta_{3} & \beta_{4}
	\end{matrix}
	\right] = \left[
	\begin{matrix}
	\alpha_{1} & \alpha_{2} & \alpha_{3} & \alpha_{4}
	\end{matrix}
	\right]\left[
	\begin{matrix}
	0 & 1 & 1 & 1\\
	1 & 0 & 1 & 1\\
	1 & 1 & 0 & 1\\
	1 & 1 & 1 & 0
	\end{matrix}
	\right]$$
	$$=\left[
	\begin{matrix}
	\alpha_{2}+\alpha_{3}+\alpha_{4} & \alpha_{1}+\alpha_{3}+\alpha_{4} & \alpha_{1}+\alpha_{2}+\alpha_{4} & \alpha_{1}+\alpha_{2}+\alpha_{3}
	\end{matrix}
	\right]$$
	所以,另一组基为:
	$$\beta_{1} = \alpha_{2}+\alpha_{3}+\alpha_{4} = \left[
	\begin{matrix}
	0 & 1\\
	1 & 1\\
	\end{matrix}
	\right]$$
	$$
	\beta_{2} = \alpha_{1}+\alpha_{3}+\alpha_{4} = \left[
	\begin{matrix}
	1 & 0\\
	1 & 1\\
	\end{matrix}
	\right]
	$$
	$$
	\beta_{3} = \alpha_{1}+\alpha_{2}+\alpha_{4} = \left[
	\begin{matrix}
	1 & 1\\
	0 & 1\\
	\end{matrix}
	\right]
	$$
	$$
	\beta_{4} = \alpha_{1}+\alpha_{2}+\alpha_{3} = \left[
	\begin{matrix}
	1 & 1\\
	1 & 0\\
	\end{matrix}
	\right]
	$$
	(2)\\
	矩阵在$\alpha_{1},\alpha_{2},\alpha_{3},\alpha_{4}$下的坐标为$(0,1,2,-3)^{T}$.在$\beta_{1},\beta_{2},\beta_{3},\beta_{4}$下坐标为
	$$A^{-1}(0,1,2,-3)^{T} = (0,-1,-2,3)^{T}
	$$
\end{proof}
\noindent 8.
\begin{proof}[解]
	令
	$$e_{1} = \left[
	\begin{matrix}
	1 & 0\\
	0 & 0\\
	\end{matrix}
	\right],e_{2} = \left[
	\begin{matrix}
	0 & 1\\
	0 & 0\\
	\end{matrix}
	\right],e_{3} = \left[
	\begin{matrix}
	0 & 0\\
	1 & 0\\
	\end{matrix}
	\right], e_{4} = \left[
	\begin{matrix}
	0 & 0\\
	0 & 1\\
	\end{matrix}
	\right]$$
	则,$e_{1},e_{2},e_{3},e_{4}$为$R^{2\times 2}$的一组基,且
	$$
	\left[
	\begin{matrix}
	\alpha_{1} & \alpha_{2} & \alpha_{3} & \alpha_{4}
	\end{matrix}
	\right] = \left[
	\begin{matrix}
	e_{1} & e_{2} & e_{3} & e_{4}
	\end{matrix}
	\right]\left[
	\begin{matrix}
	1 & 1 & 1 & 1\\
	0 & 1 & 1 & 1\\
	0 & 0 & 1 & 1\\
	0 & 0 & 0 & 1
	\end{matrix}
	\right] = \left[
	\begin{matrix}
	e_{1} & e_{2} & e_{3} & e_{4}
	\end{matrix}
	\right]M_{1}
	$$
	$$
	\left[
	\begin{matrix}
	\beta_{1} & \beta_{2} & \beta_{3} & \beta_{4}
	\end{matrix}
	\right] = \left[
	\begin{matrix}
	e_{1} & e_{2} & e_{3} & e_{4}
	\end{matrix}
	\right]\left[
	\begin{matrix}
	-1 & 1 & 1 & 1\\
	1 & -1 & 1 & 1\\
	1 & 1 & -1 & 1\\
	1 & 1 & 1 & -1
	\end{matrix}
	\right] = \left[
	\begin{matrix}
	e_{1} & e_{2} & e_{3} & e_{4}
	\end{matrix}
	\right]M_{2}
	$$
	显然$M_{1},M_{2}$都可逆,且该线性空间维数为4,所以这两组向量都是基.\\
	令$\alpha = \left[
	\begin{matrix}
	a_{0} & a_{1}\\
	a_{2} & a_{3}\\
	\end{matrix}
	\right]$\\
	$\alpha$在$e_{1},e_{2},e_{3},e_{4}$下的坐标为$\left[
	\begin{matrix}
	a_{0} \\
	a_{1}\\
	a_{2} \\
	a_{3}
	\end{matrix}
	\right]$\\
	所以,$\alpha$在$\alpha_{1},\alpha_{2},\alpha_{3},\alpha_{4}$下的坐标为
	$$
	M_{1}^{-1}\left[
	\begin{matrix}
	a_{0} \\
	a_{1}\\
	a_{2} \\
	a_{3}
	\end{matrix}
	\right] = \left[
	\begin{matrix}
	a_{0}-a_{1} \\
	a_{1}-a_{2}\\
	a_{2}-a_{3} \\
	a_{3}
	\end{matrix}
	\right]
	$$
	(2)\\
	$$
	\left[
	\begin{matrix}
	\beta_{1} & \beta_{2} & \beta_{3} & \beta_{4}
	\end{matrix}
	\right]
	=
	\left[
	\begin{matrix}
	e_{1} & e_{2} & e_{3} & e_{4}
	\end{matrix}
	\right]M_{2}
	$$
	
	$$
	= \left[
	\begin{matrix}
	\alpha_{1} & \alpha_{2} & \alpha_{3} & \alpha_{4}
	\end{matrix}
	\right]M_{1}^{-1}M_{2} = 
	\left[
	\begin{matrix}
	\alpha_{1} & \alpha_{2} & \alpha_{3} & \alpha_{4}
	\end{matrix}
	\right]
	\left[
	\begin{matrix}
	-2 & 2 & 0 & 0\\
	0 & -2 & 2 & 0\\
	0 & 0 & -2 & 2\\
	1 & 1 & 1 & -1
	\end{matrix}
	\right]
	$$
	即,过渡矩阵为$\left[
	\begin{matrix}
	-2 & 2 & 0 & 0\\
	0 & -2 & 2 & 0\\
	0 & 0 & -2 & 2\\
	1 & 1 & 1 & -1
	\end{matrix}
	\right]$
\end{proof}
\noindent 9.
\begin{proof}[解]
	(1)方程组通解为$k_{1}(1,0,...,-1)^{T}+k_{2}(0,1,...,-1)^{T}+...+k_{n-1}(0,0,...,1,-1)$,所以解空间的维数为$n-1$,基为$(1,0,...,-1)^{T},(0,1,...,-1)^{T},...,(0,0,...,1,-1)$\\
	(2)
	$A = \left[
	\begin{matrix}
	2 & -4 & 5 & 3\\
	3 & -6 & 4 & 2\\
	4 & -8 & 17 & 11
	\end{matrix}
	\right] \rightarrow \left[
	\begin{matrix}
	1 & -2 & -1 & -1\\
	2 & -4 & 5 & 3\\
	4 & -8 & 17 & 11
	\end{matrix}
	\right] \rightarrow \left[
	\begin{matrix}
	1 & -2 & -1 & -1\\
	0 & 0 & 7 & 5\\
	0 & 0 & 21 & 15
	\end{matrix}
	\right] \rightarrow \left[
	\begin{matrix}
	1 & -2 & -1 & -1\\
	0 & 0 & 7 & 5\\
	0 & 0 & 0 & 0
	\end{matrix}
	\right]$\\
	得到方程组通解为$k_{1}(2,1,0,0)^{T}+k_{2}(2,0,-5,7)^{T}$\\
	所以解空间维数为$2$,基为$(2,1,0,0)^{T}, (2,0,-5,7)^{T}$
\end{proof}
\noindent 10.
\begin{proof}[解]
	$$
	\left[
	\begin{matrix}
	\alpha_{1} & \cdots & \alpha_{i-1} & \alpha_{i+1} & \cdots & \alpha_{n+1}
	\end{matrix}
	\right] = \left[
	\begin{matrix}
	\alpha_{1} & \cdots & \alpha_{n}
	\end{matrix}
	\right]\left[
	\begin{matrix}
	1 & \cdots & 0 & 0 & \cdots & 0 & x_{1}\\
	\vdots & \ddots & \vdots & \vdots & \ddots & \vdots & \vdots\\
	0 & \cdots & 1 & 0 & \cdots & 0 & x_{i-1}\\
	0 & \cdots & 0 & 0 & \cdots & 0 & x_{i}\\
	0 & \cdots & 0 & 1 & \cdots & 0 & x_{1+1}\\
	\vdots & \ddots & \vdots & \vdots & \ddots & \vdots & \vdots\\
	0 & \cdots & 0 & 0 & \cdots & 1 & x_{n}\\
	\end{matrix}
	\right]
	$$
	该矩阵对应行列式值为$(-1)^{n+i}x_{i}$,不为0,即矩阵可逆,所以$\alpha_{1} , \cdots , \alpha_{i-1} , \alpha_{i+1} , \cdots , \alpha_{n+1}$线性无关,又线性空间为$n$维,所以,任取$n$个向量都是一组基.\\
	$$\alpha_{n+1} = x_{1}\alpha_{1}+x_{2}\alpha_{2}+...+x_{n}\alpha_{n}$$
	$$\frac{1}{x_{1}}\alpha_{n+1} = \alpha_{1}+\frac{x_{2}}{x_{1}}\alpha_{2}+...+\frac{x_{n}}{x_{1}}\alpha_{n}$$
	$$\alpha_{1} = -\frac{x_{2}}{x_{1}}\alpha_{2}-...-\frac{x_{n}}{x_{1}}\alpha_{n} + \frac{1}{x_{1}}\alpha_{n+1}$$
	所以$\alpha_{1}$在这组基下坐标为$[-\frac{x_{2}}{x_{1}}, -\frac{x_{3}}{x_{1}},..., -\frac{x_{n}}{x_{1}},\frac{1}{x_{1}}]^{T}$
\end{proof}
\noindent 13.
\begin{proof}[解]
	$c_{1}c_{3} \neq 0$,所以$\alpha = -\frac{1}{c_{1}}(c_{2}\beta+c_{3}\gamma)$\\
	$\forall v \in span(\alpha,\beta), \exists \lambda_{1}, \lambda_{2} \in R, v=\lambda_{1}\alpha+\lambda_{2}\beta $\\
	$v = (\lambda_{2} - \frac{c_{2}\lambda_{1}}{c_{1}})\alpha-\frac{c_{3}\lambda_{1}}{c_{1}}\gamma \in span(\alpha, \gamma)$\\
	$span(\alpha, \beta) \subseteq span(\alpha,\gamma)$\\
	同理可证$span(\alpha, \beta) \supseteq span(\alpha,\gamma)$\\,得到$span(\alpha, \beta) = span(\alpha,\gamma)$
\end{proof}
\noindent 15.
\begin{proof}[解]
	(1).\\
	充分性:\\
	令$B=[\beta_{1}, \beta_{2},...,\beta_{n}]$,则$C(B) = span(\beta_{1},...,\beta_{n})$.\\
	而$C(B) \subseteq N(A)$.\\
	所以$\forall \beta_{i} \in C(B), \beta_{i} \in N(A)$\\
	即$A\beta_{i} = 0$,也就是说$A\beta_{1} = 0,..., A\beta_{n} = 0$.\\\
	所以$AB=A[\beta_{1},...,\beta_{n}] = 0$.\\
	必要性:\\
	$AB=0$,则$A[\beta_{1},...,\beta_{n}] = 0$\\
	即$\forall \beta_{i} \in C(B), A\beta_{i} = 0, \beta_{i} \in N(A)$\\
	所以$C(B) \subseteq N(A)$.充要性得证.\\
	(2).\\
	由(1)得,$AB=0 \Rightarrow C(B) \subseteq N(A)$\\
	所以$dim(C(B)) \leq dim(N(A))$\\
	而$N(A)$就是$Ax=0$的解空间,所以$dim (N(A)) = n - r(A)$\\
	所以$r(B) = dim (C(B)) \leq dim(N(A)) = n- r(A)$,即$r(A)+r(B) \leq n$
\end{proof}
\noindent 16.
\begin{proof}[解]
	$W_{1}+W_{2} = span(\alpha_{1}, \alpha_{2}, \beta_{1}, \beta_{2})$\\
	$[\alpha_{1}, \alpha_{2}, \beta_{1}, \beta_{2}] = \left[
	\begin{matrix}
	2 & 0 & 1 & 1\\
	0 & -2 & 1 & -3\\
	1 & 1 & 0 & 2\\
	3 & 5 & -1 & 0\\
	-1 & -3 & 1 & 5
	\end{matrix}
	\right] \rightarrow \left[
	\begin{matrix}
	2 & 0 & 1 & 0\\
	0 & -2 & 1 & 0\\
	0 & 0 & 0 & 1\\
	0 & 0 & 0 & 0\\
	0 & 0 & 0 & 0
	\end{matrix}
	\right]$\\
	所以$W_{1}+W_{2}$的维数为3,一组基为$\alpha_{1},\alpha_{2},\beta_{2}$.\\
	根据维数公式$dim (W_{1} \cap W_{2}) = dim W_{1} + dim W_{2} - dim (W_{1}+W_{2}) = 1$\\
	注意到$\beta_{1} \in W_{2}, \beta_{1} = \frac{1}{2}\alpha_{1} - \frac{1}{2}\alpha_{2} \in W_{1}$,所以$\beta_{1} \in W_{1} \cap W_{2}$,即$\beta_{1}$可以作为$W_{1} \cap W_{2}$的一组基
\end{proof}
\noindent 18.
\begin{proof}[解]
	(1)当$A=A_{1}$时,\\
	$<\alpha,\beta> = \alpha^{T}A_{1}\beta = a_{1}b_{1}+2a_{2}b_{2}+a_{3}b_{3} = \beta^{T}A_{1}\alpha = <\beta, \alpha>$\\
	$<k\alpha, \beta> = k\alpha^{T}A_{1}\beta = k<\alpha,\beta>$\\
	$<\alpha_{1} + \alpha_{2}, \beta> = (\alpha_{1} + \alpha_{2})^{T}A_{1}\beta = (\alpha_{1}^{T} + \alpha_{2}^{T})A_{1}\beta = \alpha_{1}^{T}A_{1}\beta + \alpha_{2}^{T}A_{1}\beta = <\alpha_{1}, \beta>+<\alpha_{2}, \beta>$\\
	$<\alpha, \alpha> = a_{1}^{2}+2a_{2}^{2}+a_{3}^{2} \geq 0$恒成立,且当且仅当$\alpha = 0$时,$<\alpha,\alpha> = 0$\\
	所以,当$A=A_{1}$时为内积.\\
	(2)当$A=A_{2}$时,\\
	$<\alpha,\alpha> = a_{1}^{2}+a_{2}^{2} - 2a_{1}a_{2}+2a_{1}a_{3} = (a_{1}-a_{2})^{2}+2a_{1}a_{3}$\\
	取$a_{1} = a_{2} > 0, a_{3} < 0$,则$<\alpha,\alpha> < 0$.\\
	所以,当$A=A_{2}$时不为内积.\\
	(3)当$A=A_{3}$时,\\
	$<\alpha, \beta> = a_{1}b_{1}-a_{2}b_{1}+a_{3}b_{1}-a_{1}b_{2}+2a_{2}b_{2}+3a_{1}b_{3}+3a_{3}b_{3}$\\
	$<\beta, \alpha> = a_{1}b_{1}-a_{2}b_{1}+3a_{3}b_{1} -a_{1}b_{2}+2a_{2}b_{2}+a_{1}b_{3}+3a_{3}b_{3}$\\
	$<\alpha, \beta> \neq <\beta, \alpha>$\\
	所以,当$A=A_{3}$时不为内积.\\
\end{proof}
\noindent 19.
\begin{proof}[证]
	(1)$<\beta, \alpha> = b_{1}a_{1} - b_{2}a_{1}-b_{1}a_{2}+3b_{2}a_{2} = a_{1}b_{1}-a_{2}b_{1}-a_{1}b_{2}+3a_{2}b_{2} = <\alpha, \beta>$\\
	(2)$<k\alpha, \beta> = ka_{1}b_{1}-ka_{2}b_{1}-ka_{1}b_{2}+3ka_{2}b_{2} = k(a_{1}b_{1}-a_{2}b_{1}-a_{1}b_{2}+3a_{2}b_{2}) = k<\alpha, \beta>$\\
	(3)取$\gamma = [c_{1},c_{2}]^{T}$, $<\alpha+\beta, \gamma> = (a_{1}+b_{1})c_{1}-(a_{2}+b_{2})c_{1}-(a_{1}+b_{1})c_{2}+3(a_{2}+b_{2})c_{2} =a_{1}c_{1}-a_{2}c_{1}-a_{1}c_{2}+3a_{2}c_{2} + b_{1}c_{1}-b_{2}c_{1}-b_{1}c_{2}+3b_{2}c_{2} = <\alpha, \gamma> + <\beta, \gamma>$\\
	(4)$<\alpha, \alpha> = a_{1}^{2}-a_{2}a_{1}-a_{1}a_{2}+3a_{2}^{2} = a_{1}^{2}-2a_{1}a_{2}+3a_{2}^{2} = (a_{1}-a_{2})^{2}+2a_{2}^{2} \geq 0$,当且仅当$a_{1} = a_{2} = 0$时$<\alpha, \alpha> = 0$.\\
	综上,这是$R^{2}$的一个内积.
\end{proof}
\noindent 21.
\begin{proof}[证]
	假设$\beta_{1},\beta_{2},\alpha_{1},\alpha_{2},...,\alpha_{r}$线性相关,则存在不全为$0$的数$l_{1},l_{2},k_{1},k_{2},...,k_{r}$,使得\\
	$$l_{1}\beta_{1} + l_{2}\beta_{2} + k_{1}\alpha_{1}+k_{2}\alpha_{2}+...+k_{r}\alpha_{r} = 0$$
	首先$l_{1},l_{2}$不全为0,否则$\alpha_{1},...,\alpha_{r}$线性相关,矛盾.\\
	在等式两边与$l_{1}\beta_{1}+l_{2}\beta_{2}$做内积.得
	$$<l_{1}\beta_{1}+l_{2}\beta_{2}, l_{1}\beta_{1} + l_{2}\beta_{2} + k_{1}\alpha_{1}+k_{2}\alpha_{2}+...+k_{r}\alpha_{r}> = 0$$
	$$<l_{1}\beta_{1}+l_{2}\beta_{2},l_{1}\beta_{1}+l_{2}\beta_{2}> = 0$$
	$$l_{1}\beta_{1}+l_{2}\beta_{2} = 0$$
	所以$\beta_{1},\beta_{2}$线性相关,矛盾.\\
	所以,$\beta_{1},\beta_{2},\alpha_{1},\alpha_{2},...,\alpha_{r}$线性无关.
\end{proof}
\noindent 22.
\begin{proof}[证]
	$\alpha_{1},\alpha_{2}$已经正交,只需标准化.\\
	$$q_{1} = \frac{\alpha_{1}}{||\alpha_{1}||} = \frac{1}{\sqrt{2}}[1,0,1,0]^{T}$$
	$$q_{2} = \frac{\alpha_{2}}{||\alpha_{2}||} = \frac{1}{\sqrt{5}}[0,1,0,2]^{T}$$
	设$\beta = [x_{1},x_{2},x_{3},x_{4}]^{T} \in R^{4}$,且$\beta \bot \alpha_{1}, \beta \bot \alpha_{2}$\\
	得到
	$$
	\left\{
	\begin{aligned}
	x_{1}+x_{3} = 0\\
	x_{2}+2x_{4} = 0
	\end{aligned}
	\right.
	$$
	得到基础解系为
	$$
	\alpha_{3} = [-1,0,1,0]^{T}, \alpha_{4} = [0,-2,0,1]^{T}
	$$
	已正交,只需标准化.
	$$q_{3} = \frac{\alpha_{3}}{||\alpha_{3}||} = \frac{1}{\sqrt{2}}[-1,0,1,0]^{T}$$
	$$q_{4} = \frac{\alpha_{4}}{||\alpha_{4}||} = \frac{1}{\sqrt{5}}[0,-2,0,1]^{T}$$
	所以$q_{1},q_{2},q_{3},q_{4}$为$R^{4}$的一组标准正交基.
\end{proof}
\noindent 24.
\begin{proof}[证]
	(1)必要性:\\
	令
	$$A = [\alpha_{1}, ..., \alpha_{n}], B = [\beta_{1},...,\beta_{n}]$$
	则$A,B$为正交矩阵,$A^{T}A = I_{n}, B^{T}B = I_{n}$\\
	令过渡矩阵为$M$,即$B=AM$,则$B^{T} = M^{T}A^{T}$\\
	相乘有$I_{n} = B^{T}B=M^{T}A^{T}AM = M^{T}M$,所以$M$为正交矩阵.\\
	(2)充分性:\\
	$$B^{T}B = M^{T}A^{T}AM = M^{T}M = I_{n}$$
	所以$\beta_{1},\beta_{2},...,\beta_{n}$为标准正交基.
\end{proof}
\noindent 25.
\begin{proof}[证]
	记方程组为$Ax=b$,$A = \left[
	\begin{matrix}
	1 & 1\\
	2 & -1\\
	-2 & 4\\
	\end{matrix}
	\right], b = \left[
	\begin{matrix}
	1\\
	2\\
	7
	\end{matrix}
	\right]$
	$$
	A^{T}A=\left[
	\begin{matrix}
	1 & 2 & -2\\
	1 & -1 & 4
	\end{matrix}
	\right]\left[
	\begin{matrix}
	1 & 1\\
	2 & -1\\
	-2 & 4
	\end{matrix}
	\right] = \left[
	\begin{matrix}
	9 & -9\\
	-9 & 18
	\end{matrix}
	\right]
	$$
	$$
	(A^{T}A)^{-1} = \left[
	\begin{matrix}
	2/9 & 1/9\\
	1/9 & 1/9
	\end{matrix}
	\right]
	$$
	最小二乘解为
	$$x = (A^{T}A)^{-1}A^{T}b = \left[
	\begin{matrix}
	2/9 & 1/9\\
	1/9 & 1/9
	\end{matrix}
	\right]\left[
	\begin{matrix}
	1 & 2 & -2\\
	1 & -1 & 4
	\end{matrix}
	\right]\left[
	\begin{matrix}
	1\\
	2\\
	7
	\end{matrix}
	\right] = \left[
	\begin{matrix}
	2/9 & 1/9\\
	1/9 & 1/9
	\end{matrix}
	\right]\left[
	\begin{matrix}
	-9\\
	27
	\end{matrix}
	\right] = \left[
	\begin{matrix}
	1\\
	2
	\end{matrix}
	\right]$$
\end{proof}
\noindent 26.
\begin{proof}[解]
	$[A|b] \rightarrow \left[
	\begin{matrix}
	1 & 2 & -1 & -1\\
	2 & 0 & 1 & 1\\
	2 & -4 & 2 & 1\\
	4 & 0 & 0 & -2
	\end{matrix}
	\right] \rightarrow \left[
	\begin{matrix}
	1 & 2 & -1 & -1\\
	0 & 4 & -3 & -3\\
	0 & 8 & -4 & -3\\
	0 & 8 & -4 & -2
	\end{matrix}
	\right] \rightarrow \left[
	\begin{matrix}
	1 & 2 & -1 & -1\\
	0 & 4 & -3 & -3\\
	0 & 0 & 2 & 3\\
	0 & 0 & 0 & 1
	\end{matrix}
	\right]$\\
	$rank(A) < rank(A|b)$,所以$Ax=b$无解.\\
	$$A^{T}A = \left[
	\begin{matrix}
	1 & 2 & 2 & 4\\
	2 & 0 & -4 & 0\\
	-1 & 1 & 2 & 0
	\end{matrix}
	\right]\left[
	\begin{matrix}
	1 & 2 & -1\\
	2 & 0 & 1\\
	2 & -4 & 2\\
	4 & 0 & 0
	\end{matrix}
	\right] = \left[
	\begin{matrix}
	25 & -6 & 5\\
	-6 & 20 & -10\\
	5 & -10 & 6
	\end{matrix}
	\right]$$
	$$(A^{T}A)^{-1} = \left[
	\begin{matrix}
	5/96 & -7/192 & -5/48\\
	-7/192 & 125/384 & 55/96\\
	-5/48 & 55/96 & 29/24
	\end{matrix}
	\right]$$
	最小二乘解为\\
	$$x = (A^{T}A)^{-1}A^{T}b = \left[
	\begin{matrix}
	-11/24\\
	25/48\\
	23/12
	\end{matrix}
	\right]$$
\end{proof}
\noindent 28.
\begin{proof}[解]
	$\left[
	\begin{matrix}
	2 & 1 & -1 & 1 & -3\\
	1 & 1 & -1 & 0 & 1 
	\end{matrix}
	\right] \rightarrow \left[
	\begin{matrix}
	1 & 0 & 0 & 1 & -4\\
	0 & 1 & -1 & -1 & 5
	\end{matrix}
	\right]$\\
	得到解空间的一组基为$(0,1,1,0,0)^{T},(-1,1,0,1,0)^{T},(4,-5,0,0,1)^{T}$\\
	将$\alpha_{1},\alpha_{2},\alpha_{3}$Schmidt正交化,即令\\
	$$
	q_{1} = \frac{\alpha_{1}}{||\alpha_{1}||} = \frac{1}{\sqrt{2}}(0,1,1,0,0)^{T}
	$$
	$$
	q_{2} = \frac{\alpha_{2}-q_{1}q_{1}^{T}\alpha_{2}}{||\alpha_{2}-q_{1}q_{1}^{T}\alpha_{2}||} = \frac{1}{\sqrt{10}}(-2,1,-1,2,0)^{T}
	$$
	$$
	q_{3} = \frac{\alpha_{3}-q_{1}q_{1}^{T}\alpha_{3}-q_{2}q_{2}^{T}\alpha_{3}}{||\alpha_{3}-q_{1}q_{1}^{T}\alpha_{3}-q_{2}q_{2}^{T}\alpha_{3}||} = \frac{1}{3\sqrt{35}}(7,-6,6,13,5)^{T}
	$$
\end{proof}
\noindent 29.
\begin{proof}[解]
	令$A=[\alpha_{1},\alpha_{2}]$,则$C(A)^{\perp} = N(A^{T}) = \{x|A^{T}x=0\}$\\
	得到齐次方程组\\
	$$
	\left\{
	\begin{aligned}
	x_{1}+x_{2}+2x_{3}=0\\
	x_{1}+x_{3}=0
	\end{aligned}
	\right.
	$$
	得到一个基础解系为$\alpha_{3} = [-1,-1,1]^{T}$\\
	所以$C(A)^{\perp} = span(\alpha_{3})$\\
	$[\alpha_{1},\alpha_{2},\alpha_{3},\alpha] = \left[
	\begin{matrix}
	1 & 1 & -1 & 3\\
	1 & 0 & -1 & 2 \\
	2 & 1 & 1 & 1
	\end{matrix}
	\right] \rightarrow \left[
	\begin{matrix}
	1 & 1 & -1 & 3\\
	0 & 1 & 0 & 1 \\
	0 & 0 & 3 & -4
	\end{matrix}
	\right]$\\
	解得唯一解$(2/3,1,-4/3)^{T}$,所以,$\alpha = \frac{2}{3}\alpha_{1}+\alpha_{2}-\frac{4}{3}\alpha_{3}$\\
	其中$\frac{2}{3}\alpha_{1}+\alpha_{2} \in C(A)$, $-\frac{4}{3}\alpha_{3} \in C(A)^{T}$\\
	所以,$\alpha$在$C(A)$中的正交投影为
	$$\frac{2}{3}\alpha_{1}+\alpha_{2} = [5/3,2/3,7/3]^{T}$$
	在$C(A)^{\perp}$中的投影为
	$$-\frac{4}{3}\alpha_{3} = [4/3,4/3,-4/3]^{T}$$
\end{proof}
\noindent 31.
\begin{proof}[解]
	令$\epsilon_{1},\epsilon_{2},\epsilon_{3}$到$\eta_{1},\eta_{2},\eta_{3}$的过渡矩阵为$M$.\\
	则$$[\eta_{1},\eta_{2},\eta_{3}] = [\epsilon_{1},\epsilon_{2},\epsilon_{3}]M$$
	$$M=[\epsilon_{1},\epsilon_{2},\epsilon_{3}]^{-1}[\eta_{1},\eta_{2},\eta_{3}]$$
	$$=\left[
	\begin{matrix}
	8 & -16 & 9\\
	-6 & 7 & -3 \\
	7 & -13 & 7
	\end{matrix}
	\right]^{-1}\left[
	\begin{matrix}
	1 & 3 & 2\\
	-2 & -1 & 1 \\
	1 & 2 & 2
	\end{matrix}
	\right] = \left[
	\begin{matrix}
	1 & 1 & -3\\
	1 & 2 & -5 \\
	1 & 3 & -6
	\end{matrix}
	\right]$$
	所以,$T$在$\eta_{1},\eta_{2},\eta_{3}$下的表示矩阵为
	$$
	M^{-1}AM=\left[
	\begin{matrix}
	1 & 2 & 2\\
	3 & -1 & -2 \\
	2 & -3 & 1
	\end{matrix}
	\right]
	$$
\end{proof}
\noindent 33.
\begin{proof}[解]
	令$\epsilon_{1},\epsilon_{2},\epsilon_{3}$到$\eta_{1},\eta_{2},\eta_{3}$的过渡矩阵为$M$.\\
	则$$[\eta_{1},\eta_{2},\eta_{3}] = [\epsilon_{1},\epsilon_{2},\epsilon_{3}]M$$
	$$T[\epsilon_{1},\epsilon_{2},\epsilon_{3}] = [T\epsilon_{1},T\epsilon_{2},T\epsilon_{3}] = [\eta_{1},\eta_{2},\eta_{3}] = [\epsilon_{1},\epsilon_{2},\epsilon_{3}]M$$
	所以,$T$在$\epsilon_{1},\epsilon_{2},\epsilon_{3}$下的表示矩阵为$M$.\\
	$T$在$\eta_{1},\eta_{2},\eta_{3}$下的表示矩阵为$M' = M^{-1}MM = M$\\
	$$M=[\epsilon_{1},\epsilon_{2},\epsilon_{3}]^{-1}[\eta_{1},\eta_{2},\eta_{3}]$$
	$$ = \left[
	\begin{matrix}
	1 & 2 & 1\\
	0 & 1 & 1 \\
	1 & 0 & 1
	\end{matrix}
	\right]^{-1}\left[
	\begin{matrix}
	1 & 2 & 2\\
	2 & 2 & -1 \\
	-1 & -1 & -1
	\end{matrix}
	\right] = \left[
	\begin{matrix}
	-2 & -3/2 & 3/2\\
	1 & 3/2 & 3/2 \\
	1 & 1/2 & -5/2
	\end{matrix}
	\right]$$
\end{proof}
\end{document}